% This file is part of the Data Analysis Recipes project.
% Copyright 2011 2012 David W. Hogg (NYU), Jo Bovy (IAS), and Dustin Lang (Princeton)

\documentclass[12pt,twoside]{article}
\usepackage{amssymb,amsmath,mathrsfs,../hogg_endnotes,natbib}

%%Figure caption
\makeatletter
\newsavebox{\tempbox}
\newcommand{\@makefigcaption}[2]{%
\vspace{10pt}{#1.--- #2\par}}%
\renewcommand{\figure}{\let\@makecaption\@makefigcaption\@float{figure}}
\makeatother

\setlength{\emergencystretch}{2em}%No overflow

\newcommand{\notenglish}[1]{\textsl{#1}}
\newcommand{\aposteriori}{\notenglish{a~posteriori}}
\newcommand{\apriori}{\notenglish{a~priori}}
\newcommand{\adhoc}{\notenglish{ad~hoc}}
\newcommand{\etal}{\notenglish{et al.}}
\newcommand{\eg}{\notenglish{e.g.}}

\newcommand{\documentname}{document}
\newcommand{\sectionname}{Section}
\newcommand{\equationname}{equation}
\newcommand{\problemname}{Exercise}
\newcommand{\solutionname}{Solution}
\newcommand{\notename}{note}

\newcounter{problem}
\newenvironment{problem}{\paragraph{\problemname~\theproblem:}\refstepcounter{problem}}{}
\newenvironment{comments}{\paragraph{\commentsname:}}{}
\newcommand{\affil}[1]{{\footnotesize\textsl{#1}}}

% matrix stuff
\newcommand{\mmatrix}[1]{\boldsymbol{#1}}
\newcommand{\inverse}[1]{{#1}^{-1}}
\newcommand{\transpose}[1]{{#1}^{\scriptscriptstyle \top}}
\newcommand{\mA}{\mmatrix{A}}
\newcommand{\mAT}{\transpose{\mA}}
\newcommand{\mC}{\mmatrix{C}}
\newcommand{\mCinv}{\inverse{\mC}}
\newcommand{\mE}{\mmatrix{E}}
\newcommand{\mQ}{\mmatrix{Q}}
\newcommand{\mS}{\mmatrix{S}}
\newcommand{\mX}{\mmatrix{X}}
\newcommand{\mY}{\mmatrix{Y}}
\newcommand{\mYT}{\transpose{\mY}}
\newcommand{\mZ}{\mmatrix{Z}}
\newcommand{\vhat}{\mmatrix{\hat{v}}}

% parameter vectors
\newcommand{\parametervector}[1]{\mmatrix{\vec{#1}}}
\newcommand{\pvtheta}{\parametervector{\theta}}

% set stuff
\newcommand{\setofall}[3]{\{{#1}\}_{{#2}}^{{#3}}}
\newcommand{\allq}{\setofall{q_i}{i=1}{N}}
\newcommand{\allx}{\setofall{x_i}{i=1}{N}}
\newcommand{\ally}{\setofall{y_i}{i=1}{N}}
\newcommand{\allxy}{\setofall{x_i,y_i}{i=1}{N}}
\newcommand{\allsigmay}{\setofall{\sigma_{yi}^2}{i=1}{N}}
\newcommand{\allC}{\setofall{\mC_i}{i=1}{N}}

% other random multiply used math symbols
\renewcommand{\d}{\mathrm{d}}
\newcommand{\like}{\mathscr{L}}
\newcommand{\Pbad}{P_{\mathrm{b}}}
\newcommand{\Ybad}{Y_{\mathrm{b}}}
\newcommand{\Vbad}{V_{\mathrm{b}}}
\newcommand{\bperp}{b_{\perp}}
\newcommand{\mean}[1]{\left<{#1}\right>}
\newcommand{\meanZ}{\mean{\mZ}}
\newcommand{\best}{\mathrm{best}}

% header stuff
\renewcommand{\MakeUppercase}[1]{#1}
\pagestyle{myheadings}
\renewcommand{\sectionmark}[1]{\markright{\thesection.~#1}}
\markboth{Model complexity}{}

% note stuff
\newcommand{\note}[1]{\endnote{#1}}
\def\enotesize{\normalsize}
\renewcommand{\thefootnote}{\fnsymbol{footnote}} % the ONE footnote needs this

\begin{document}
\thispagestyle{plain}\raggedbottom
\section*{Data analysis recipes:\ \\
  Comparing models of different complexity\footnotemark}

\footnotetext{The \notename s begin on page~\pageref{note:first},
  including the license\note{\label{note:first} Copyright 2011 by the
    authors.  You may copy and distribute this document provided that
    you make no changes to it whatsoever.}  and the
  acknowledgements\note{Above all we owe a debt to Sam Roweis (Toronto
    \& NYU, now deceased), who taught us everything we know about
    model selection.  In addition it is a pleasure to thank Iain
    Murray (Edinburgh) and Hans-Walter Rix (MPIA) for valuable
    comments and discussions.  This research was partially supported
    by NASA (ADP grant NNX08AJ48G), NSF (grant AST-0908357), and a
    Research Fellowship of the Alexander von Humboldt Foundation.
    This research made use of the Python programming language and the
    open-source Python packages scipy, numpy, and matplotlib.}.}

\noindent
David~W.~Hogg\\
\affil{Center~for~Cosmology~and~Particle~Physics, Department~of~Physics, New~York~University}\\
\affil{Max-Planck-Institut f\"ur Astronomie, Heidelberg}
\\[1ex]
Jo~Bovy\\
\affil{Center~for~Cosmology~and~Particle~Physics, Department~of~Physics, New~York~University}
\\[1ex]
Dustin~Lang\\
\affil{Princeton University Observatory}

\begin{abstract}
  Frequently the data analyst is faced with the comparison of models
  of very different levels of complexity or freedom to fit or explain
  the data.  We begin by discussing the measure of model complexity,
  in part to criticize the idea that it is directly related to the
  number of parameters in the model: It is not, except in rare
  circumstances.  Model selection---choosing among models of differing
  complexity---can often be avoided; we advocate avoiding it.  If it
  cannot be avoided, we discuss options.  We recommend the data-driven
  technique of leave-one-out cross-validation, which locates the model
  that best predicts left-out data.
\end{abstract}

\section{Model complexity every day}

A scientist makes model complexity decisions---implicitly or
explicitly---every day.  For example, on a trivial scale, any decision
about binning a histogram (small bins or big bins) or fitting a curve
(linear or quadratic) or combining data (average all the data taken on
the same day or all the data taken in the same year) is a decision
about model complexity; in each case the decision is about how much
freedom to give to some free function.

At a grander scale, many important scientific discoveries are model
complexity decisions.  For example, the decision that data on a star's
radial velocity as a function of time is better described by a model
that includes perturbations from an orbiting exoplanet is
simultaneously a model selection decision (a model that includes the
freedom of an exoplanet perturbation is a better description than a
model that doesn't have that freedom) and an exoplanet discovery.
That discovery is a decision to increase model complexity, but also
many scientific discoveries are decisions to decrease model
complexity; consider for example the heliocentric model of the Solar
System.  It was adopted in part because it led to an enormous decrease
in the complexity of the quantitative Solar System model, reducing
greatly the number of free parameters.

Of course a model with more freedom will, in general, lead to a better
description of the data in the trivial sense that it can reproduce the
observed features more closely.  But a model with more freedom will
also, in general, be less predictive, because finite uncertainties in
the parameters will propagate into greater uncertainty in prediction
as the number of imprecisely determined (independent) parameters
increases.  The challenge of model selection is to balance these
considerations; a scientifically good model must both describe the
existing data well and also predict new or left-out data well.\note{In
  this sense, Ockham's Razor, na\"ively construed, is wrong, or
  perhaps useless: It is not the case that the simplest model that
  explains the data is the best.  First of all, the scientist is
  (almost) never presented with two models that are \emph{equally}
  good at explaining the data.  Second of all, even if the two models
  \emph{are} equal at explaining existing data, are they equal at
  predicting new data?  Some Bayesians have taken Ockham's Razor very
  seriously, claiming that Bayesian inference and Bayesian model
  selection justifies Ockham's Razor, but this view relies on a
  careful interpretation of Ockham that doesn't necessarily do justice
  to the original.}

\section{The standard lore}

The concept of model complexity is somewhat slippery.  In the case of
a purely linear model, the model complexity is directly related to the
number of free parameters, but in every other case it is less
well-defined.  We will begin with a few examples of models of
differing or variable complexity to get at some of the issues in the
\emph{definition} of model complexity.  After that, we will turn to
methods for choosing among models of different complexity.

linear case

then consider linear components with no support at the position of the
data; do they contribute?

model selection might illuminate this.

\section{Infinite complexity}

One model that serves as an example to illustrate that model
complexity is not directly related to the \emph{number} of parameters
is the following.\note{We owe this model to Yann LeCun (NYU), via Sam
  Roweis.}  Imagine having $N$ data points $(x_i,y_i)$ in some finite
domain in $x$ and $y$, with observational uncertainties of some kind
we need not specify precisely for the moment.  Now fit to these data
the model
\begin{equation}\label{eq:lecun}
y_i = A\,\sin (k\,x + \phi)
  \quad ,
\end{equation}
where $A$ is an amplitude, $k$ is a wave number (or frequency), and
$\phi$ is a phase.  With suitable---and suitably
\emph{precise}---specification of these parameters, it is possible to
\emph{exactly fit all $N$ data points, no matter what they are}.  The
only assumption is that $x_i \ne x_j$ for $i \ne j$.  Furthermore,
there is not just one location in the $(A,k,\phi)$ space that fits all
the data points, there is an enormous infinity of solutions, no matter
how large $N$.

In the standard situation of linear fitting, if there are $K$
components---and thus $K$ free parameters---it is possible to match
\emph{exactly} at most $K$ data points, and even this is only
guaranteed if all of the components have support on the $x$ range of
the data and they are appropriately independent under the $x$-sampling
of the data.  If we equate model complexity with the power to flexibly
fit arbitrary data, then the three-parameter sinusoidal model of
\equationname~(\ref{eq:lecun}) has as much complexity as a linear
model with an \emph{infinite number of parameters}.

Of course the implied paradox has a simple resolution: In order to fit
the data precisely, the parameters of the model---especially the wave
number $k$---must be specified to enormous \emph{precision}.  That is,
the $K$-component linear model gets its power to fit by having a large
number of parameters.  The sinusoidal model gets its power to fit by
having a large number of digits of accuracy devoted to the $k$
parameter (or, in some limits, the $A$ parameter).

Hogg:  Perhaps give a specific illustration of this here.---Hogg

When a linear model with $K=N$ is used to exactly fit all $N$ data
points, the $N$ parameters can be seen as a simple invertible linear
transformation of the data; that is, the parameters are (in some
sense) a lossless encoding of the data.  When the sinusoidal model is
used to exactly fit all $N$ points, the model again is being used to
losslessly encode the data, but now the encoding is nonlinear, and
requires of the parameters enormous precision in their representation.
This enormous precision is required because all of the information in
all $N$ data points---all the bits, if you like---must be encoded into
only three parameters.

\section{Continuously variable complexity}

Another model that serves to illustrate that model complexity is not
directly related to the number of parameters is the following, in
which there are \emph{far more parameters} than data points, but
nonetheless the model has limited freedom.  The freedom of the model
is controlled by a continuous smoothness parameter, which effectively
makes this a model with continuously variable complexity.

Imagine a set of $N$ data points $(x_i,y_i)$ which deviate from a
$K$-dimensional (or $K$-parameter) linear model by the addition (to
the $y$ values) of noise samples $e_i$ drawn from Gaussian
distributions of (presumed known) variances $\sigma_{yi}^2$.  The
model can be written as
\begin{equation}
y_i = \sum_{j=1}^K
  a_{ij}\,x_j + e_i
  \quad ,
\end{equation}
or, in matrix notation,
\begin{equation}
\mY = \mA\,\mX + \mE
  \quad ,
\end{equation}
where $\mY$ is an $N$-dimensional vector of the $y_i$, $\mA$ is an
$N\times K$ matrix of (presumed known) coefficients $a_{ij}$, $\mX$ is
a $K$-dimensional vector of the parameters $x_j$, and $\mE$ is an
$N$-dimensional vector of the noise draws or errors $e_i$.  The
standard thing to do is to minimize a $\chi^2$ scalar
\begin{equation}
\chi^2 = \sum_{i=1}^N
  \frac{\left[y_i - a_{ij}\,x_j\right]^2}{\sigma_{yi}^2}
  \quad ,
\end{equation}
where this is scientifically justified when all the aforementioned
assumptions (linear model space includes the truth, noise is Gaussian
with known variances, and so on) are true.  This objective ($\chi^2$)
is truly a scalar in the sense that it has a compact matrix
representation
\begin{equation}
\chi^2 = \transpose{\left[\mY-\mA\,\mX\right]}
  \,\mCinv\,\left[\mY-\mA\,\mX\right]
  \quad .
\end{equation}
In this situation, when $N<K$, the best-fit value for the parameters
(the elements of $\mX$) is given by
\begin{equation}
\mX_\best = \inverse{\left[\mAT\,\mCinv\,\mA\right]}
  \,\left[\mAT\,\mCinv\,\mY\right]
  \quad ,
\end{equation}
which was found by forcing the full derivative of the $\chi^2$ scalar
to vanish.  In this formulation, if the (errors of the) $y_i$ are
independent, the covariance matrix $\mC$ is the diagonal $N\times N$
matrix with the noise variances $\sigma_{yi}^2$ on the diagonal.

Frequently (for engineers, if not astronomers) the condition $N<K$
does not hold, but there are smoothness expectations that provide
enough support to set the parameters nonetheless.  The simplest
implementation of a smoothness prior or regularization is the addition
to $\chi^2$ of a quadratic penalty for derivatives in the parameters
$x_j$.  In this picture, we imagine the parameters $x_j$ are ordered
such that nearby $j$ are nearby in the appropriate sense.  The scalar
objective gets modified in this case to
\begin{equation}
\chi_r^2 = \sum_{i=1}^N
  \frac{\left[y_i - a_{ij}\,x_j\right]^2}{\sigma_{yi}^2}
  + \epsilon\,\sum_{j=1}^{K-1}\left[x_{j+1}-x_j\right]^2
  \quad ,
\end{equation}
where $\epsilon$ is a control parameter that controls the smoothness
of the model, or the stiffness of it, or else the model complexity in
some sense.  Under this objective, the best fit parameter vector is
given by
\begin{equation}
\mX_\best = \inverse{\left[\mAT\,\mCinv\,\mA + \epsilon\,\mQ\right]}
  \,\left[\mAT\,\mCinv\,\mY\right]
  \quad ,
\end{equation}
where the $K\times K$ matrix $\mQ$ is a tri-diagonal matrix that looks
like this
\begin{equation}
\mQ = \left[\begin{array}{cccccc}
    1 &-1 & 0 & 0 & 0 & 0 \\
   -1 & 2 &-1 & 0 & 0 & 0 \\
    0 &-1 & 2 &-1 & 0 & 0 \\
    0 & 0 &-1 & 2 &-1 & 0 \\
    0 & 0 & 0 &-1 & 2 &-1 \\
    0 & 0 & 0 & 0 &-1 & 1 \\
  \end{array}\right]
  \quad ,
\end{equation}
in the case of $K=6$ parameters and generalizes in the obvious way for
different $K$.

\section{Standard decision criteria}

AIC, BIC, and why they are all inapplicable in general.

\section{Bayesian decision theory}

Why it is \emph{so sensitive} to the priors.

\section{Cross-validation}

Why it rocks; Sam quotation.

\section{No need to decide?}

Mix your models whenever you can; or carry them all forward.

\clearpage
\markright{Notes}\theendnotes

\clearpage
\begin{thebibliography}{}\markright{References}
\bibitem[Bovy, Hogg, \& Roweis(2009)]{bovy}
  Bovy,~J., Hogg,~D.~W., \& Roweis, S.~T., 2009,
  Extreme deconvolution: inferring complete distribution functions from noisy, heterogeneous, and incomplete observations, 
  arXiv:0905.2979 [stat.ME]
\end{thebibliography}

\end{document}
