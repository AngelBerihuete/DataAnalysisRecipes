% This file is part of the Data Analysis Recipes project.
% Copyright 2016 David W. Hogg (NYU).

% to-do
% -----
% - First draft
% - Make sure the ``you'' in the tables is returned to as an issue; who is the audience?
% - Career-stage considerations.
% - deal with all HOGG DWH or all-caps.

% style notes
% -----------
% - These documents are ``Chapters'', no?  Use \documentname.
% - [LaTeX] newline after every full stop for proper git diffing
% - [LaTeX] eqnarray not equation
% - \note{} right after word if you are endnoting a phrase,
%   but after the full stop if endnoting the sentence or paragraph.

\documentclass[12pt,twoside,pdftex]{article}
\usepackage{dar_endnotes}
\usepackage{enumitem}

\newcommand{\chaptertitle}{The costs and benefits of releasing your data and open-sourcing your code}
\newcommand{\shorttitle}{Releasing data and open-sourcing code}
\usepackage{amssymb}
\usepackage{amsmath}
\usepackage{mathrsfs}
\usepackage{natbib}
\usepackage{float}
\usepackage{graphicx}

% Figure captions
\makeatletter
\newsavebox{\tempbox}
\newcommand{\@makefigcaption}[2]{%
\vspace{10pt}{#1.--- #2\par}}%
\renewcommand{\figure}{\let\@makecaption\@makefigcaption\@float{figure}}
\makeatother

% No overflow
\setlength{\emergencystretch}{2em}

% Spacing -- Bring the Bringhurst
\linespread{1.08} % nearly 10/13
\setlength{\parindent}{1.08\baselineskip} % nearly square
\frenchspacing

% Typography
\newcommand{\documentname}{\textsl{Chapter}}
\newcommand{\sectionname}{Section}
\newcommand{\equationname}{equation}
\newcommand{\project}[1]{\textsl{#1}}
\newcommand{\foreign}[1]{\textsl{#1}}
\newcommand{\code}[1]{\texttt{\detokenize{#1}}}
\newenvironment{pseudocode}{\begin{ttfamily}\detokenize\obeylines}{}
\newcommand{\aposteriori}{\foreign{a~posteriori}}
\newcommand{\apriori}{\foreign{a~priori}}
\newcommand{\adhoc}{\foreign{ad~hoc}}
\newcommand{\etal}{\foreign{et al.}}
\newcommand{\eg}{\foreign{e.g.}}
\newcommand{\vs}{\foreign{vs.}}
\newcommand{\affil}[1]{{\footnotesize\textsl{#1}}}
\newcommand{\acronym}[1]{{\small{#1}}}

% Notes
\newcommand{\notename}{note}
\newcommand{\note}[1]{\endnote{#1}}
\def\enotesize{\normalsize}
\renewcommand{\thefootnote}{\fnsymbol{footnote}} % the ONE footnote needs this

% Problems
\newcommand{\problemname}{Problem}
\newcounter{problem}
\newenvironment{problem}{\paragraph{\problemname~\theproblem:}%
\refstepcounter{problem}}{}

% Math stuff
\newcommand{\mmatrix}[1]{\boldsymbol{#1}}
\newcommand{\inverse}[1]{{#1}^{-1}}
\newcommand{\transpose}[1]{{#1}^{\scriptscriptstyle \top}}
\newcommand{\parametervector}[1]{\mmatrix{#1}}
\newcommand{\bvec}[1]{\mmatrix{#1}}
\newcommand{\setof}[1]{\{{#1}\}}
\newcommand{\dd}{\mathrm{d}}
\newcommand{\given}{\,|\,}
\newcommand{\mean}[1]{\left<{#1}\right>}

% Headers
\renewcommand{\MakeUppercase}[1]{#1}
\pagestyle{myheadings}
\renewcommand{\sectionmark}[1]{\markright{\thesection.~#1}}
\markboth{\shorttitle}{}

% References.
\usepackage{color,hyperref}
\definecolor{linkcolor}{rgb}{0,0,0.5}
\hypersetup{colorlinks=true,linkcolor=linkcolor,citecolor=linkcolor,
            filecolor=linkcolor,urlcolor=linkcolor}
\newcommand{\arxiv}[1]{\href{http://arxiv.org/abs/#1}{\textsl{arXiv}:#1}}
\newcommand{\doi}[1]{\href{http://doi.org/#1}{\textsc{doi}:#1}}
\newcommand{\isbn}[1]{\textsc{isbn:}#1}

\newlength{\myl}
\settowidth{\myl}{9.\rule{\itemsep}{0pt}}
\setlist{nosep, leftmargin=\myl}

\begin{document}\sloppy\sloppypar\raggedbottom\thispagestyle{plain}%
%
\section*{Data analysis recipes:\\ \chaptertitle\footnotemark}
%
\footnotetext{%
    The notes begin on page~\pageref{note:first}, including the
    license\note{\label{note:first}
        Copyright 2016 the author.
        This work is licensed under a
        \href{http://creativecommons.org/licenses/by-nc-nd/3.0/deed.en\_US}{%
            Creative Commons Attribution-NonCommercial-NoDerivs 3.0 Unported
            License}.}
    and the acknowledgements\note{%
        I would like to thank
          Alice Allen (ASCL),
          Michael Blanton (NYU),
          Dustin Lang (Toronto),
          Dan Foreman-Mackey (UW),
          Adrian Price-Whelan (Princeton),
          Hans-Walter Rix (MPIA),
          Sam Roweis (deceased),
          Jake Vanderplas (UW),
          and the entire astrophysics community
        for valuable advice, ideas, and vision over the years.
        I also thank
          Eric Agol (UW),
          Joss Bland-Hawthorn (Sydney),
          Kai Blin (DTU),
          Caroline Bot (Strasbourg),
          C.~Titus Brown (Davis),
          Daniel Cotton (UNSW),
          Kyle Cranmer (NYU),
          Casey Greene (Penn),
          Robert T.~McGibbon (D.~E.~Shaw),
          Paul McMillan (Lund),
          Andy Mueller (NYU),
          Will O'Mullane (ESAC),
          Philip Stark (Berkeley),
          and
          Jason Wright (PSU)
        for comments on some of the specific content in this document.
        This project benefitted from various research grants, especially
          from NSF, NASA,
          and the The Moore--Sloan Data Science Environment.}
}

\noindent
\textbf{David~W.~Hogg}\\
\affil{Simons~Center~for~Data~Analysis,
       Simons~Foundation}\\
\affil{Center~for~Cosmology~and~Particle~Physics, Department~of~Physics,
       New~York~University}\\
\affil{Center~for~Data~Science,
       New~York~University}\\
\affil{Max-Planck-Institut~f\"ur~Astronomie, Heidelberg}

\begin{abstract}
The sharing of data and code brings an investigator great benefits
(such as greater impact, good will, and reproducibility), but it also
incurs substantial costs (such as documentation and support
requirements, and risks of getting scooped).
The decision to share or not to share data and code is always
technically subjective, depending---as it does---both on
posterior beliefs about the world and on a utility function.
This \documentname\ does not contain a manifesto that advocates sharing!
It is a discussion of the considerations that must be weighed with one's
beliefs and utility in order to make a sensible decision.
A scientist's utility function is likely to be a strong function of personality and career
stage; a scientist's beliefs (about, say, colleagues and funding
agencies) will be a strong function of field of study;
there is no simple conclusion.
However, benefits outnumber (not necessarily \emph{outweigh}, but \emph{outnumber})
costs for code sharing, and the opposite may be true for data sharing;
that is, it is more likely that an investigator will want to share
code than data.
However there are some ``economies of scale'' in which simultaneous
sharing of code and data can be better than sharing either alone.
In general, investigators working in fields that are fecund---where there
are more results to be investigated than there are investigators---are
more incentivized to share than those working in fields that are more
sterile.
\end{abstract}~ \\

\section{Subjectivity, utility, pragmatism}

As we have discussed elsewhere in this series of
\documentnames\note{We discuss this in \documentnames\ not yet
  written, about model selection, and the presentation of
  results. These are \emph{decisions}, which for our purposes are
  events in which we have to depart from probabilistic
  reasoning---or bring it to a close---and commit to a single, irreversible outcome.}, good
decision-making requires both quantitative beliefs (in the form of
posterior pdfs over outcomes) and a quantitative utility function.
That's a high bar!
But even reasonably sensible decisions require some thinking about
what things are reasonably probable to happen, and what their costs or
value to you might be.
Nowhere is this more stark than in the decisions that every scientist
faces about the sharing of their data and data-analysis code with
their colleagues and the world at large.

Since we are not going to get quantitative in what follows---it is
notoriously hard to develop quantitative beliefs about things that
actually \emph{matter}\note{It has always struck me as odd and
  terrible that we can obtain extremely principled, quantitative
  posterior beliefs about, say, the orbital parameters of a binary
  star, but we don't even try when it comes to things like whether an
  employee is a good hire, or a project is worth starting. This has
  many causes, not the least of which is a lack of good data, a point
  to which I will return.}---I won't say too much more about either
your posterior beliefs or your utility, except for two things:
The first is that we should be focused on the \emph{long-term} future,
not the short-term future.\note{As I (hope I) discuss in more detail
  in other \documentname s in this series, a sensible framework for
  thinking about utility in decision-making is your long-term future
  discounted free cash flow (LTFDFCF). This is a great utility if you
  imagine marginal fungibility between preferences and cash. That is,
  you don't have to imagine that money can buy you love, but you do
  have to imagine that money can buy you a few more minutes of quality
  time with your loved ones. The difference between cash flow and
  profit is relevant. The ideas of ``long-term'' and ``discounted''
  are mentioned a bit in the main text following this \notename. I
  learned about the LTFDFCF in the short period in which I consulted
  for the company \project{Google}, which (in 2008-ish, anyway) had
  adopted it explicitly as it's own objective, and (furthermore, in
  this context) identified \emph{opportunity costs} as its largest
  running category of cost!}
Now what constitutes the ``long term'' for a graduate student is
likely to be very different from what constitutes the long term for a
tenured faculty member; for the former it might be two years, and for
the latter it might be twenty; the term is deliberately vague.
In economics, it is defined (roughly) to be the timescale on which
all factors of production and costs can vary.
For a scientist, it could be defined (roughly) to be the timescale on
which what you are working on, how you work on it (and why) could all
vary.
That is, the long term is exactly the time period over which you can't
make confident predictions about outcomes!
However, we want to build careers, not results in papers, and to build
a career (or a community, or anything of lasting importance), you want
to be focusing on the long term.

The second (and related) point is that there is an effective
\emph{discount rate}.
That is---all other things being equal---you prefer immediate gains to
future gains, and the relationship is exponential in time.
This discount rate is \emph{also} a strong function of career stage
(and probably personality); it is likely to be much higher for a
graduate student than a tenured faculty member\note{Until the tenured
  faculty member is diagnosed with a terminal illness!}
These considerations will be relevant to anyone weighing the costs and
benefits in what follows:
Some of these costs and benefits are paid and received immediately,
and some take time to build up; for instance the long-term
preservation of your data might be a cost you will only really feel
many years in the future, while you might need to write documentation
immediately.
These costs, therefore, ought to be weighed differently by different
scientists at different career stages.

HOGG: Going to treat data and code in parallel for most of what follows.

HOGG: The huge issue that there is almost no DATA to point to. This is
all qualitative crap. Cite what IS relevant.

\section{Overview}

Most of the points of discussion that will appear in this
\documentname\ appear in summary form in \tablename~\ref{tab:data} and
\tablename~\ref{tab:code}.%
\newlength{\cwidth}\setlength{\cwidth}{0.483\textwidth}%
\begin{table}%
\begin{tabular}{@{}p{\cwidth}|p{\cwidth}@{}}% note @{} entries to remove space on edges
\multicolumn{2}{c}{\textbf{releasing data}} \\ \hline
\multicolumn{1}{c|}{costs} & \multicolumn{1}{c}{benefits} \\ \hline
\begin{enumerate}\raggedright
\item you might get scooped
\item you will have to document the data and its issues
\item you will have to answer questions about the data, sometimes dumb ones
\item you will have to beat down bad and incorrect papers written about your data
\item your reputation could suffer if the data are used wrongly
\item you will have to pay for archiving and web interface
\item you will have to make plans for long-term stewardship
\item there might be legal and privacy issues
\end{enumerate}&\begin{enumerate}\raggedright
\item more science gets done, more papers get written; all measures of impact increase
\item results become (more) reproducible and hackable
\item you are motivated to write up analyses and discoveries in a timely manner
\item you are motivated to document the data
\item outsiders reanalyze and generate new ideas for calibration or analysis; these improve your own future work
\item data are preserved
\item you get cred and visibility and build community
\item you are motivated to produce data products on some schedule
\item it's the right thing to do
\end{enumerate}\end{tabular}
\caption{Summary of costs and benefits for data release, vaguely
  ordered by importance (from most important to least). All of these
  telegraphic points are discussed in more depth in the main text. The
  word ``you'' is used loosely here---different readers will be at
  different career stages and have very different utilities---but this
  will be elaborated upon in the main text.\label{tab:data}}
\end{table}

\begin{table}%
\begin{tabular}{@{}p{\cwidth}|p{\cwidth}@{}}% note @{} entries to remove space on edges
\multicolumn{2}{c}{\textbf{open-sourcing code}} \\ \hline
\multicolumn{1}{c|}{costs} & \multicolumn{1}{c}{benefits} \\ \hline
\begin{enumerate}\raggedright
\item you might get scooped
\item you can be embarrassed by your ugly code, or by your failure to comment or document
\item you will have to answer questions (and often stupid or irrelevant ones) about the code, and spend time documenting
\item you will have to consider pull requests or otherwise maintain the code, possibly far into the future
\item you might have to beat down bad and incorrect results generated with your code
\item your reputation could suffer if the code is used wrongly
\item there could be legal (including military) and license issues
\end{enumerate}&\begin{enumerate}\raggedright
\item more science gets done, more papers get written; all measures of impact increase
\item you get motivation to clean and document the code
\item results become (more) reproducible and (more) trustworthy
\item outsiders find bugs or make improvements to your code and deliver pull requests
\item you get cred and visibility and build community
\item citation rates might go up
\item code is preserved
\item code becomes searcheable (including by you!) and \emph{backed up}
\item there are good sites for long-term archiving and interface
\item can establish priority on an idea or prior art on a method
\end{enumerate}\end{tabular}
\caption{Summary of costs and benefits for open-source code release,
  vaguely ordered from most important to least. As with
  \tablename~\ref{tab:data}, all of these are discussed in more depth
  in the main text, including the subjective ``you''. Some of the
  issues listed here apply not just to open-sourcing code, but also to
  releasing code, publishing code, or shipping
  software.\label{tab:code}}
\end{table}

These \tablename s make an attempt to boil down the (relatively nuanced, I
hope) discussion in what follows into a set of quotable bullet points.
Also, the points in the \tablename s are organized somehow by my
perception of their ``importance'' or weight for a typical scientist,
where ``typical'' is defined (I guess) to be ``just like me''.
This is \emph{not} the order in which I discuss them in what follows.
In what follows, I attempt to discuss them in an order that is (in
some sense) political or sociological or anthropological.
That is, I try to situate them somewhat in terms of power, status,
career stage, and so on.

HOGG: Once again, there are no DATA. Sucks!

\section{Personal considerations}

Search! Hosting! Preservation!

Documentation burden (bad). Documentation pressure (good).

Exposure of the horror (bad). Encouragement to be less horrible (good).

HOGG: how releasing code and data together might help a bit?

\section{Team and management considerations}

HOGG: how releasing code and data together helps a lot!

\section{Proprietary period and timeline}

\section{Ethical and legal considerations}

You might expect me here to go all ``ethically, you must release your
data!''; I won't, because there are many situations in which it might
be in fact \emph{unethical} to release data and code.
That said, in most cases, releasing your data and code \emph{does}
make the world a better place: You are sharing your ideas and effort
with the world, and empowering others with less to do more.
As I said in the outset, I think scientists should make decisions that
optimize their utility, and that their utility should be based on a
long-term view.
In the long term, everyone---and scientists in particular---benefit
from behaving ethically; in our business you \emph{can} do well by
doing good.
So, other things being equal, the ethical considerations usually fall
on the side of public release of data and code, especially after some
kind of proprietary period or perhaps with some conditions that
protect the people on your team who worked so hard to get those data
and build that code (especially the young ones).
Here is some more discussion of those places where the ethical
considerations are a bit more complicated:

Ownership and licensing (next section).

IANAL, but there are lawyers! But NumFOCUS and the like.

Patents.

Military restrictions; military uses. HOGG Cite SM!

Coming back to the personal point about exposing the horror and
requirements to document: Both of these have ethical aspects: The more
we expose the horror, the more we make people comfortable with
sharing. The more we document, the more we help others with
reproducibility and with new science.

\section{Licensing}

Open-sourcing your code means more than just publishing it.

Again IANAL.

Licensing has not really been explored for data, at least not in most
scientific domains.

\clearpage
\markright{Notes}\theendnotes

\clearpage
\raggedright
\begin{thebibliography}{}\markright{References}
\bibitem[Foreman-Mackey \etal(2012)]{emcee}
  Foreman-Mackey,~D., Hogg,~D.~W., Lang,~D., \& Goodman,~J.\ 2012,
  \project{emcee}: The \acronym{MCMC} Hammer,
  \textit{Pubs.\ Astron.\ Soc.\ Pac.}\ \textbf{125} 306--312
  \doi{10.1086/670067}
  (also \arxiv{1202.3665}).
\end{thebibliography}

\end{document}
