% This file is part of the Data Analysis Recipes project.
% Copyright 2013 the author.

% to-do
% -----
% - write
% - make examples

\documentclass[12pt,twoside]{article}
%%Figure caption
\makeatletter
\newsavebox{\tempbox}
\newcommand{\@makefigcaption}[2]{%
\vspace{10pt}{#1.--- #2\par}}%
\renewcommand{\figure}{\let\@makecaption\@makefigcaption\@float{figure}}
\makeatother

\newcommand{\exampleplot}[1]{%
\begin{center}%
\includegraphics[width=0.5\textwidth]{#1}%
\end{center}%
}
\newcommand{\exampleplottwo}[2]{%
\begin{center}%
\includegraphics[width=0.5\textwidth]{#1}%
\includegraphics[width=0.5\textwidth]{#2}%
\end{center}%
}

\setlength{\emergencystretch}{2em}%No overflow

\newcommand{\notenglish}[1]{\textsl{#1}}
\newcommand{\aposteriori}{\notenglish{a~posteriori}}
\newcommand{\apriori}{\notenglish{a~priori}}
\newcommand{\adhoc}{\notenglish{ad~hoc}}
\newcommand{\etal}{\notenglish{et al.}}
\newcommand{\eg}{\notenglish{e.g.}}

\newcommand{\documentname}{document}
\newcommand{\sectionname}{Section}
\newcommand{\equationname}{equation}
\newcommand{\figurenames}{\figurename s}
\newcommand{\problemname}{Exercise}
\newcommand{\problemnames}{\problemname s}
\newcommand{\solutionname}{Solution}
\newcommand{\notename}{note}

\newcommand{\note}[1]{\endnote{#1}}
\def\enotesize{\normalsize}
\renewcommand{\thefootnote}{\fnsymbol{footnote}} % the ONE footnote needs this

\newcounter{problem}
\newenvironment{problem}{\paragraph{\problemname~\theproblem:}\refstepcounter{problem}}{}
\newcommand{\affil}[1]{{\footnotesize\textsl{#1}}}

% matrix stuff
\newcommand{\mmatrix}[1]{\boldsymbol{#1}}
\newcommand{\inverse}[1]{{#1}^{-1}}
\newcommand{\transpose}[1]{{#1}^{\scriptscriptstyle \top}}
\newcommand{\mA}{\mmatrix{A}}
\newcommand{\mAT}{\transpose{\mA}}
\newcommand{\mC}{\mmatrix{C}}
\newcommand{\mCinv}{\inverse{\mC}}
\newcommand{\mQ}{\mmatrix{Q}}
\newcommand{\mS}{\mmatrix{S}}
\newcommand{\mX}{\mmatrix{X}}
\newcommand{\mY}{\mmatrix{Y}}
\newcommand{\mYT}{\transpose{\mY}}
\newcommand{\mZ}{\mmatrix{Z}}
\newcommand{\vhat}{\mmatrix{\hat{v}}}

% parameter vectors
\newcommand{\parametervector}[1]{\mmatrix{#1}}
\newcommand{\pvtheta}{\parametervector{\theta}}

% set stuff
\newcommand{\setofall}[3]{\{{#1}\}_{{#2}}^{{#3}}}
\newcommand{\allq}{\setofall{q_i}{i=1}{N}}
\newcommand{\allx}{\setofall{x_i}{i=1}{N}}
\newcommand{\ally}{\setofall{y_i}{i=1}{N}}
\newcommand{\allxy}{\setofall{x_i,y_i}{i=1}{N}}
\newcommand{\allsigmay}{\setofall{\sigma_{yi}^2}{i=1}{N}}
\newcommand{\allS}{\setofall{\mS_i}{i=1}{N}}

% other random multiply used math symbols
\renewcommand{\d}{\mathrm{d}}
\newcommand{\mean}[1]{\left<{#1}\right>}
\newcommand{\like}{\mathscr{L}}


% header stuff
\renewcommand{\MakeUppercase}[1]{#1}
\pagestyle{myheadings}
\renewcommand{\sectionmark}[1]{\markright{\thesection.~#1}}
\markboth{Inferring distributions from noisy measurements}{}

\newcommand{\data}{D}
\newcommand{\pars}{\theta}
\newcommand{\cv}{\mathrm{CV}}
\newcommand{\pcv}{p_{\cv}}

\begin{document}
\thispagestyle{plain}\raggedbottom
\section*{Data analysis recipes:\ Inferring distributions \\from noisy measurements\footnotemark}

\footnotetext{The \notename s begin on page~\pageref{note:first},
  including the license\note{\label{note:first} Copyright 2013 by the
    author.  You may copy and distribute this document provided that
    you make no changes to it whatsoever.}  and the
  acknowledgements\note{It is a pleasure to thank
      Jo Bovy (IAS),
      Dustin Lang (CMU),
      Dan Foreman-Mackey (NYU),
      Hans-Walter Rix (MPIA),
      Sam Roweis (deceased), and
      Bernhard Sch\"olkopf (MPI-IS)
    for valuable comments and discussions.  This research was
    partially supported by NASA (ADP grant NNX08AJ48G), NSF (grant
    AST-0908357), and a Research Fellowship of the Alexander von
    Humboldt Foundation.  This research made use of the Python
    programming language and the open-source Python packages scipy,
    numpy, and matplotlib.}.}

\noindent
David~W.~Hogg\\
\affil{Center~for~Cosmology~and~Particle~Physics, Department~of~Physics, New~York~University}\\
\affil{Max-Planck-Institut f\"ur Astronomie, Heidelberg}
%% \\[1ex]
%% Jo~Bovy\\
%% \affil{Institute for Advanced Study, Princeton}
%% \\[1ex]
%% Dustin~Lang\\
%% \affil{Princeton University Observatory}\\
%% \affil{Department of Physics, Carnegie Mellon University}

\begin{abstract}
Frequently we have some repeated measurement of a quantity $Q$, one
measurement of $Q$ per object in some sample of objects; we often want
to know the distribution of this quantity in the population from which
the objects are drawn.  The simplest idea is to make a histogram of
the measurements, or estimates, of the quantity.  The measurements are
necessarily noisy, so this empirical histogram of estimates is always
broader or less informative than the true distribution for $Q$.  If
each measurement of $Q$ is based on a likelihood function that
includes an accurate noise model---even in the case that the noise is
different for every measurement---it is possible to replace the simple
histogramming with a hierarchical inference that obtains a
marginalized likelihood (or posterior probability) for the true
distribution of $Q$ that \emph{would have been observed} in a sample
with much less noise per measurement.  The hierarchical inference is a
kind of deconvolution, because it infers a more informative
distribution from less-informative samples---capitalizing on the
distribution information in the set of all samples---but it proceeds
by forward-modeling the data generation process.  We provide some
examples and more strong advice: Investigators sometimes create even
\emph{less} informative (broader) proxies for the distribution than
the empirical histogram of estimates; all such methods are bad.
\end{abstract}

In astronomy, there is a traditional idea that ``bad data are worse
than no data''.  I grew up (academically) hearing this phrase over and
over.  Now, if there is a theme to the \documentnames\ that make up
this series, it \emph{could} be that ``a lot of bad data is as good as
a small amount of good data''.  This theme will justified by the
contents of this \documentname; of course the theme is only factually
correct when the bad data meet certain requirements, especially about
our knowledge---or ability to obtain knowledge---about the noise
contributions that make the bad data bad.

\section{population statistics}

Imagine you have measured the eccentricities of the orbits of a bunch
of exoplanets, or measured the lengths of a large number of fish, or
even tested a group of patients for whether they are carying a
specific disease.  If you are studying the population as a whole, you
want to know as best as possible the \emph{true distribution} of the
quantity you are measuring or testing for.  If your measurements are
perfect (or extremely low in noise) then you don't need to do anything
sophisticated: Make a histogram of the measured eccentricities or
lengths, or compute the fraction that test positive for the disease,
and you are done.  The problem is, in many important situations, your
data just aren't this good.\note{Hogg: Rutherford quotation about
  statistics and getting better data, reference, comment.  Ack BS.}

As an introduction to our problem, let's continue with the medical
example: Imagine that we have a pretty good test for a particular
disease, such that it's false-positive and false negative rates are
percent-level.  To get very specific, let's discuss these rates in
terms of a \emph{likelihood}\note{Hogg: likelihood name rant.}, or a
probability that the test reports a positive (a ``yes''), symbolized
by the binary indicator $\hat{Q}=1$ or a negative (a ``no''),
symbolized by $\hat{Q}=0$, conditional on the patient's true state, of
either actually having the disease (``sick''), symbolized by the
binary indicator $Q=1$, or not having the disease (``well''),
symbolized by $Q=1$.  The likelihood ``function'' is here just a
$2\times 2$ matrix of probabilities $p(\hat{Q}\given Q)$ in our
standard probability notation:\note{Hogg: cite probability calculus
  document here and in references.}
\begin{equation}\label{eq:testlike}
\mbox{\begin{tabular}{rc|cc}
\multicolumn{2}{l|}{$p(\hat{Q}\given Q)$}
               & ``yes'' & ``no'' \\
             & & $\hat{Q}=1$ & $\hat{Q}=0$ \\
\hline
``sick'' & $Q=1$ & $0.97$ & $0.03$ \\
``well'' & $Q=0$ & $0.05$ & $0.95$
\end{tabular}}
\end{equation}
This test is a good one, but (of course) not perfect.  It is not clear
how any doctor or researcher could accurately know the four numbers in
the likelihood matrix~(\ref{eq:testlike}), but that is a subject
beyond the scope of this \documentname.\note{Hogg!}

Now imagine that an epidemiologist has a fair sample of 1000 people
from some population and wants to know the rate of this disease in the
population.  He or she administers the test to the subjects and finds
that 56 of the 1000 people test positive (get ``yes'' or $\hat{Q}=1$).
What can be inferred from this measurement about the \emph{true
  rate}\note{Hogg: not responsible for any implied meaning of the word
  ``true''.}  $p(Q=1)$ of the disease (sickness) in the
population?\note{I am expressing the ``rate'' as a probability here,
  which is not too confusing, I hope.}

Since the test is pretty good, it is not crazy to consider the
absolutely simplest estimate: Perhaps the disease rate is about the
``yes'' rate, or about 54/1000 or 0.054.  (With perhaps an uncertainty
of plus or minus some error based on the binomial theorem, which would
be on the order of fifteen percent fractionally, or in absolute terms
0.007-ish.)  That estimate---0.054---is a bad estimate!  Why?  Because
the test is really expected to be wrong about five percent of the time.

An even \emph{wronger} estimate (so wrong you might imagine no-one
would ever make this mistake, but I can cite real-live
papers\note{Hogg!}) would be to say ``well, the test is wrong about
five percent of the time, so even some of those 944 `no' ($\hat{Q}=0$)
answers must be wrong, so the rate must be even \emph{higher} than
0.054.''  That's \emph{really} wrong; we will come back to its insane
wrongness below.

A much better estimate is to do what I and my coauthors consistently
advise in this series of \documentnames: Make a generative
probabilistic model---a model that gives you a probability
distribution over possible data\note{Bernhard Sch\"olkopf would tell
  us that for the model to be good for our purposes we need it to be
  not just generative but \emph{causal}.  The model should embody our
  beliefs about the causal processes that created the data.}---and do
probabilistic inference.  In this case, the simplest model has the
following two simple assumptions: \textsl{(a)}~We adopt the likelihood
given in the matrix~(\ref{eq:testlike}) above as being true, or at
least good enough.  \textsl{(b)}~We imagine that there is a true rate
or probability of sickness, call it $P_+$, in the population as a
whole.  The first assumption establishes a likelihood, and the second
establishes a prior, albeit a strange prior that itself has a
parameter, $P_+$. This rate $P_+$ is a parameter in the sense that it
is an unknown in the model.  We have hard-set the likelihood, but we
don't yet know this probability $P_+$... In Figure~?? we show the
probabilistic graphical model corresponding to these assumptions.

Now, in this generative model, what is the probability of getting a
``yes'' in the test?
\begin{eqnarray}
p(\hat{Q}=1\given P_+)
 &=& \sum_{Q=0}^1 p(\hat{Q}=1\given Q)\,p(Q\given P_+)
\\
p(\hat{Q}=1\given P_+)
 &=& p(\hat{Q}=1\given Q=1)\,P_+ + p(\hat{Q}=1\given Q=0)\,[1 - P_+]
\\
p(\hat{Q}=1\given P_+)
 &=& 0.97\,P_+ + 0.05\,[1 - P_+]
\\
p(\hat{Q}=1\given P_+)
 &=& 0.05 + 0.92\,P_+
\quad,
\end{eqnarray}
where appropriate likelihood values have been substituted in.  This is
a new likelihood, but with the prior---the parameterized prior---used
to marginalized out the health state of the individual patient.  It is
a marginalized likelihood, and can be used in inference like any other
likelihood.  This is a point we will beat to death in this
\documentname.

In fact the full data set is not one such test result but $N=1000$
such results $\hat{Q}_n$.  If we assume independence, the total
likelihood is a product, or the logarithm is a sum:
\begin{eqnarray}
D &\equiv& \setofall{\hat{Q}_n}{n=1}{N}\given P_+) 
\\
\ln p(D\given P_+) 
 &=& \sum_{n=1}^N p(\hat{Q}_n\given P_+)
\\
\ln p(D\given P_+) 
 &=& 56\,\ln p(\hat{Q}=1\given P_+) + 944\,\ln p(\hat{Q}=0\given P_+)
\\
\ln p(D\given P_+) 
 &=& 56\,\ln (0.05 + 0.92\,P_+) + 944\,\ln (0.95 - 0.92\,P_+)
\quad,
\end{eqnarray}
where a new ``full-data'' object (set, blob) $D$ is defined, the
numbers have been substituted in, and the fact that the probabilities
of $\hat{Q}$ being zero and one must sum to unity (this is a
combination of likelihoods) has been exploited.  Because this is a
full-data likelihood, we can optimize it (to get a maximum likelihood
value fo $P_+$, look at changes in the likelihood with respect to
$P_+$ to get uncertainties, or multiply by a prior PDF for $P_+$ and
compute or sample a posterior PDF... In Figure~?? the dependence of
this likelihood on $P_+$ is shown.

The maximum-likelihood rate in the population is less than 0.007!  It
is \emph{far} lower than the naive estimate of 0.056.  Why?  Because
when rates are low, healthy patients are much more likely to get
scattered into the ``yes'' category than sick patients are to be
scattered into the ``no'' category.

A few notes: If you play around with this example (and I encourage
that) you will find that what you conclude is and can be a very strong
function of what you assume about the likelihood function, which, in
real situations, you never know precisely.\note{The likelihood
  function is a quantitative description of your noise model and you
  almost never have an extremely precise model of your noise.}  The
maximum-likelihood value for $P_+$ is very close to zero, so the
``standard error'' doesn't make much sense.  This argues for switching
to a posterior sampling or a more flexible description of the
likelihood function to express your uncertainty.  The
maximum-likelihood rate $P_+$ goes to precisely zero as the number of
observed ``yes'' values goes below 50.  All these observations are
relevant and important to what follows, below.

The key idea of this introductory example is that even if your
likelihood for each object depends on latent variables or model
parameters for that object that you can't measure precisely, you can
nonetheless infer properties of the \emph{distribution} of those
latent variables.  The key idea is that through marginalization, the
likelihood can be transformed from a set of single-object likelihoods
that depend on latent variables into a full-data likelihood that
depends only on population-level variables.  These appear in the
calculation as parameters of the prior PDF, or
\emph{hyper-parameters}.  It is this meta-upping from likelihood to
marginalized likelihood, and from object parameters to population
hyper-parameters, that makes this analysis \emph{hierarchical}.

\begin{problem}
What would have been the maximum-likelihood estimate of the rate $P_+$
in the problem in this \sectionname\ if, of the 1000 patients, only 44
tested positive?  What uncertainty would you put on that rate
estimate?
\end{problem}

\begin{problem}
How much would the maximum-likelihood estimate of the rate $P+$ in the
problem in this \sectionname\ change if the likeihood values changed
by one percent?  Consider various one-percent changes.  When you make
your changes, be sure to keep the normalization correct; what
combinations of likelihood values in the likelihood matrix must sum to
unity?
\end{problem}

\begin{problem}
In many cases, even the likelihood values in the likelihood matrix
themselves are badly determined.  What should you do in this case?
For concreteness, consider a case in which the top-left value in the
matrix (currently 0.97) could be distributed anywhere between 0.99 and
0.95.  How do you re-do the analysis and what would you report for the
population rate $P_+$?
\end{problem}

\begin{problem}
In the problem in this \sectionname, the standard error on the
maximum-likelihood estimate for the population rate $P+$ included
values below zero.  What does that mean or how would you report
results in a better way than ``best-fit value plus or minus some error
bar''?
\end{problem}

\section{estimators built from estimators}

...not terrible idea: histogram of maximum-likelihood estimators.

...when can this give the right distribution?

...why will the empirical variance always be larger than the true variance?

...very bad idea: co-add likelihood functions or histogram posterior
samples.  This is like convolving your noisy data with your errors
once again!  Mistake made surprisingly often.

...concrete example

\section{hierarchical inference}

...the true disribution is your prior

...use the force (probability calculus), Luke

...how in practice to operate

...concrete example, continued from previous section

...relies strongly on good noise information.

\section{using bad data to predict good data}

...use XDQSO data as an example here.

...relies strongly on good noise information for both data sets.

...notes about the fact that accurate information about causality was
necessary to make this work.

\clearpage
\markright{Notes}\theendnotes

\clearpage
\begin{thebibliography}{}\markright{References}
\bibitem[Bovy, Hogg, \& Roweis(2009)]{bovy}
  Bovy,~J., Hogg,~D.~W., \& Roweis, S.~T., 2009,
  Extreme deconvolution: inferring complete distribution functions from noisy, heterogeneous, and incomplete observations, 
  arXiv:0905.2979 [stat.ME]
\end{thebibliography}

\end{document}
