% this file is part of the Data Analysis Recipes project
% Copyright 2012, 2013 David W. Hogg

% to-do
% -----
% - is the second-person stuff annoying?  I think it is.
% - decide about ``I'' vs ``we'' and any co-author?
% - fix intro to get to the point faster
% - swap out the DETF example, which isn't a good one
% - LTFDFCF argument is too strident
% - finish zeroth draft

% style notes
% -----------
% - ``pdf'' not ``PDF''

\documentclass[12pt,twoside,pdftex]{article}
%%Figure caption
\makeatletter
\newsavebox{\tempbox}
\newcommand{\@makefigcaption}[2]{%
\vspace{10pt}{#1.--- #2\par}}%
\renewcommand{\figure}{\let\@makecaption\@makefigcaption\@float{figure}}
\makeatother

\newcommand{\exampleplot}[1]{%
\begin{center}%
\includegraphics[width=0.5\textwidth]{#1}%
\end{center}%
}
\newcommand{\exampleplottwo}[2]{%
\begin{center}%
\includegraphics[width=0.5\textwidth]{#1}%
\includegraphics[width=0.5\textwidth]{#2}%
\end{center}%
}

\setlength{\emergencystretch}{2em}%No overflow

\newcommand{\notenglish}[1]{\textsl{#1}}
\newcommand{\aposteriori}{\notenglish{a~posteriori}}
\newcommand{\apriori}{\notenglish{a~priori}}
\newcommand{\adhoc}{\notenglish{ad~hoc}}
\newcommand{\etal}{\notenglish{et al.}}
\newcommand{\eg}{\notenglish{e.g.}}

\newcommand{\documentname}{document}
\newcommand{\sectionname}{Section}
\newcommand{\equationname}{equation}
\newcommand{\figurenames}{\figurename s}
\newcommand{\problemname}{Exercise}
\newcommand{\problemnames}{\problemname s}
\newcommand{\solutionname}{Solution}
\newcommand{\notename}{note}

\newcommand{\note}[1]{\endnote{#1}}
\def\enotesize{\normalsize}
\renewcommand{\thefootnote}{\fnsymbol{footnote}} % the ONE footnote needs this

\newcounter{problem}
\newenvironment{problem}{\paragraph{\problemname~\theproblem:}\refstepcounter{problem}}{}
\newcommand{\affil}[1]{{\footnotesize\textsl{#1}}}

% matrix stuff
\newcommand{\mmatrix}[1]{\boldsymbol{#1}}
\newcommand{\inverse}[1]{{#1}^{-1}}
\newcommand{\transpose}[1]{{#1}^{\scriptscriptstyle \top}}
\newcommand{\mA}{\mmatrix{A}}
\newcommand{\mAT}{\transpose{\mA}}
\newcommand{\mC}{\mmatrix{C}}
\newcommand{\mCinv}{\inverse{\mC}}
\newcommand{\mQ}{\mmatrix{Q}}
\newcommand{\mS}{\mmatrix{S}}
\newcommand{\mX}{\mmatrix{X}}
\newcommand{\mY}{\mmatrix{Y}}
\newcommand{\mYT}{\transpose{\mY}}
\newcommand{\mZ}{\mmatrix{Z}}
\newcommand{\vhat}{\mmatrix{\hat{v}}}

% parameter vectors
\newcommand{\parametervector}[1]{\mmatrix{#1}}
\newcommand{\pvtheta}{\parametervector{\theta}}

% set stuff
\newcommand{\setofall}[3]{\{{#1}\}_{{#2}}^{{#3}}}
\newcommand{\allq}{\setofall{q_i}{i=1}{N}}
\newcommand{\allx}{\setofall{x_i}{i=1}{N}}
\newcommand{\ally}{\setofall{y_i}{i=1}{N}}
\newcommand{\allxy}{\setofall{x_i,y_i}{i=1}{N}}
\newcommand{\allsigmay}{\setofall{\sigma_{yi}^2}{i=1}{N}}
\newcommand{\allS}{\setofall{\mS_i}{i=1}{N}}

% other random multiply used math symbols
\renewcommand{\d}{\mathrm{d}}
\newcommand{\mean}[1]{\left<{#1}\right>}
\newcommand{\like}{\mathscr{L}}


% header stuff
\renewcommand{\MakeUppercase}[1]{#1}
\pagestyle{myheadings}
\renewcommand{\sectionmark}[1]{\markright{\thesection.~#1}}
\markboth{Making decisions}{Introduction}

\begin{document}
\thispagestyle{plain}\raggedbottom
\section*{Data analysis recipes:\\
  Making decisions}

\footnotetext{%
  The \notename s begin on page~\pageref{note:first}, including the
  license\note{\label{note:first}%
    Copyright 2012, 2013 David W. Hogg (NYU).  Right now this document
    (though publicly available) is NOT FOR DISTRIBUTION.  Eventually,
    and once it is ready, I will release it with the license ``You may
    copy and distribute this document provided that you make no
    changes to it whatsoever''.}
  and the acknowledgements\note{%
    It is a pleasure to thank
      Jo Bovy (IAS),
      Dan Foreman-Mackey (NYU),
      Dustin Lang (CMU),
      Phil Marhall (Oxford), and
      Sam Roweis (deceased)
    for discussions and comments.  This
    research was partially supported by the US National Aeronautics
    and Space Administration and National Science Foundation.}}

\noindent
David~W.~Hogg\\
\affil{Center~for~Cosmology~and~Particle~Physics, Department~of~Physics, New York University}\\
and~\affil{Max-Planck-Institut f\"ur Astronomie, Heidelberg}
%% \\[1ex]
%% A. Nother Author
%% \affil{Oxford University}

\begin{abstract}
Probabilistic inference at its best gives probabilistic information
about parameters and models.  In many cases of importance---as with
making a discovery or experimental design or purchasing or delivery of
results---it is necessary to make a hard decision or choice.  It is
almost never the right move to choose the \emph{most probable} model
or option; it is rare that the most probable is the option that
delivers the most benefit to the decider.  Instead the decider ought
to specify a utility function and work to optimize expected utility,
or minimize expected loss, or minimize maximal loss.  In real-world
situations of building experiments or operating an organization, to be
generally useful the utility must be specified in (the equivalent of)
currency (dollar) units.  Whether the decision is about which
observation to make next, which of several hypotheses to reject, or
what to write in a paper title and abstract, the employed utility
function should be an approximation to the decider's \emph{long-term
  future discounted free cash flow}.
\end{abstract}

\section{You never decide on probability alone}

Imagine you have taken imaging of a patch of the sky and you want to
know the brightness of a particular star, or you have performed a
survey of consumers and you want to know what fraction of them prefer
a particular product.  In each case you have noisy data, which
contains noisy information about the quantity you care about.  The
scientifically responsible thing to report is the full likelihood
function or the posterior probability distribution
function\endnote{Because the computation of a posterior pdf involves
  additional assumptions over and above the assumptions made in
  computing a likelihood, if you release only posterior pdfs, you must
  always \emph{also} report the prior pdf you employed.  I discuss
  this more in other contributions in this series.} (pdf) for the
quantity of interest---the star brightness or the consumer fraction.
Only these full functions properly transmit the full information
content of the measurement.

Nonetheless, in various situations, it is necessary to deliver \emph{a
  particular value} for the quantity of interest.  For example, you
might be calculating a telescope exposure time (for your next
measurement) or deciding how many units of a product to ship to a
store.  In each case, you have to use \emph{some value} to perform the
calculation.  Well, strictly speaking, you don't \emph{have} to use a
single value to do this calculation, but in practice it is often the
only practicable way.  For example, sometimes you have a superior
officer who demands a number, or a referee who insists that you put a
number in the abstract of your paper.  When you are in this situation,
what number should you report or use?  The rules of probabilistic
inference \emph{do not answer this question}.  The rules of
probabilistic inference only tell you how to manipulate probabilities
and frequencies for data and parameters.  For example, they don't tell
you whether to report the mode, mean, or median of the posterior pdf
or the maximum of the likelihood or some percentile upper or lower
limit.  (And they \emph{certainly} don't tell you how to adjust if the
likelihood or posterior is peaked somewhere that seems likely to get
you fired, demoted, ridiculed, or bankrupt.\note{Some would argue that
  you are \emph{not allowed} to modify what you say about your results
  in order to avoid embarrassment or equivalent problems.  But in
  fact, if you look inside, you realize you do it all the time!  When
  your result is perfectly in line with previous results, you give the
  central value and an error bar.  When your result is much higher,
  you might give the central value and the error bar, but you will
  \emph{also} give the probability interval corresponding to the
  overlap with the previous result.  This is not just \emph{done} but
  \emph{essential}.  It would be irresponsible \emph{not} to modify
  how you deliver results in response to concerns about believability,
  sensibleness, consistency, and impact.})

Similarly, in discovery or detection: Do you take ``three sigma'' or
``five sigma'' or ``ten'' to be the threshold?  Probabilistic
inference does not answer this question either.\note{cite MODEL
  SELECTION \documentname} If it is bad for you to report the
existence of something that isn't there, you will set your threshold
high.  If it is important that you \emph{not miss} something that
\emph{is} there, you will set it low.  As these examples illustrate,
it is your \emph{utility}---what is useful to you or important to
you---that informs your decision, not just the results of the
probabilistic inference.

Most importantly, if you are trying to decide \emph{between models} or
\emph{between hypotheses}, and these models have different posterior
probabilities (possibly marginalized over all parameters) how do you
decide?  Once again, the rules of probabilistic inference do not tell
you which model to choose.\note{There is a huge literature and many
  methods devoted to computing ``Bayes factors'' or fully marginalized
  likelihoods or posterior probabilities for parameterized models.
  This is all very valuable work, but if you take that massive
  computational effort and then just choose the \emph{more probable
    model}, you are probably making a mistake.}  You can choose the
more probable model, but that is rarely \emph{required}.  It is often
not even sensible: What if model $A$ is slightly more probable, but
requires thousands of CPU-years to compute, whereas model $B$ is
slightly less probable but you can compute it on your mobile phone?
For many purposes, model $B$ would be the model to choose; it depends
on the \emph{use} to which the model is being put.  Economic factors
(and we will get very specific about what those factors are) must play
a role.

Finally, and as we have argued elsewhere\note{where?}, if you can
\emph{possibly avoid} making a decision you should.  The laws of
probabilistic inference tell you how to \emph{mix} the models or
marginalize over unknowns; if you can live with a mixture you should.
Only decide when you have no choice, as when you have to put a number
in an abstract, type a line in a published table or chart, or cut
metal in an experiment or instrument.  It is only then---when
irreversible action is imminent---that choice is \emph{really}
required.

\section{You have a scalar objective}

Given that the world only provides probabilistic or noisy information,
the fact that any rational actor (think ``brilliant robot''\note{The
  rational-inference robot is a very useful piece of
  thought-experiment equipment introduced by \citet{jaynes}, ch.~1 and
  thereafter.})  can make any hard choice---decide anything---results
from the fact that he or she (or henceforth ``it'') has some kind of
objective function that it is optimizing.  Even a Bayesian actor that
chooses to report from a posterior pdf the most probable parameter
values is effectively adopting ``probability'' as it's objective.  A
frequentist actor that chooses to report the maximum-likelihood values
is effectively adopting ``likelihood'' as its objective.  For general
situations, both of these choices of objective is absurd; I will
return to that below.\note{More soon, but briefly it is easy to see
  that the likelihood is a bad objective: The likelihood can be
  maximal in a location in parameter space you know to be ruled out by
  theoretical considerations or else previous experiments.}

...You must have an objective... probabilists can only optimize the
\emph{expected value} (or other posterior statistic of) that
objective... frequentists are even \emph{more} restricted: They can
only optimize for the best-case or worst-case scenario in the ``bag of
models''.

...Might want to maximize completeness \emph{and} minimize
false-positive rate... All objectives conflict at some point...

...Concept of the convex sum... What is that?..

...Note that for many investigators there is an ambiguity between
prior and utility; they are used similarly in
decision-making... Indeed there is a product ambiguity between them...

\section{Money must be involved}

% The DETF example doesn't bring to light the most important issue;
% after all, the existence of any single mission creates an exchange
% rate for marginal exchanges between money and DETF.  As for the
% different missions taking different times, that is about the
% discount rate, not about money \notenglish{per se}.

Back in 2005-ish, the cosmology community underwent a soul-searching
exercise about future projects intended to investigate the nature of
\emph{dark energy}, or the mechanical cause of the late-time
acceleration of the Universe's expansion.  The Dark-Energy Task
Force (DETF) ended up recommending that projects be ranked on the
basis of their ability to constrain two dark-energy parameters, $w_0$
and $w_a$.  In particular, they argued that the relevant scalar or
utility or figure of merit (FoM) to consider is the inverse of the
area inside the 95-percent confidence interval (implicitly a
likelihood contour) in the $w_0$--$w_a$ plane.\note{\cite{detf}.}

In many ways this was a breakthrough: A committee recommends a
quantitative method for comparing (necessarily expensive, complex, and
lengthy) projects, and describes that quantitative method in terms of
merit or utility.  For this the DETF deserves to be commended.
Furthermore, the proposed FoM scalar relates to the informaton content
in the data delivered by the experiments, so it is eminently sensible.

However, there is an equally valid point of view in which this
recommendation was \emph{useless}.  The reason is that although the
FoM permits the community to compare projects in terms of information
delivered about dark energy, no two projects deliver the same FoM, no
two projects have the same total cost, and no two projects deliver
results at the same point in the future.  If one project delivers an
FoM of 1200, costs $8\times 10^6$~USD, and takes four years, and
another delivers an FoM of 1950 and costs $17\times 10^6$~USD, and
takes 3.5 years, how are we to compare them?  We need to be able to
understand the trade-offs between money, FoM, and time.  And sure
enough, although the DETF was able to classify projects into
categories (FoM bins), it was not able to rank the relative value of
very qualitatively different projects.

...similarly spectroscopic fibers in \project{SDSS}...

...Automated decision making for follow-up...

...Objective experimental design...

Fundamentally this currency requirement flows from the point that
there are always many \emph{qualitatively different choices}.  Even in
a decision as trivial as what to report as the result of a
measurement: You don't just have to decide what value to report, you
also have to decide whether you are going to \emph{just} report a best
value, or a best value and an error bar, or a best value and a set of
confidence intervals, or an upper limit.  Each of these choices
requires different amounts of ink and different caveats and
effectively stakes different claims.  Effectively, they come at
different costs and bring different benefits.  They can only be
compared if they can be put on the same footing; that footing is
fundamentally economic.

\section{Utility calculations}

% coordinate this with the above ``scalar objective'' section

...there are two issues: How to compute your utility for some outcome,
and then what to do with those computations given a probabilistic
inference or prediction...

...minimax...

...posterior utility maximization\note{There is an excellent---if
  idiosyncratic---discussion of decision-making in a probabilistic
  context in \cite{jaynes}, chs.~13 and 14.}...

\section{Why is posterior probability a bad objective?}

...first a reminder of why likelihood is a bad objective...

...the idea of a narrow high peak and a very wide lower peak...  This
is the problem solved by marginalization (Bayes evidence) but then you
have to decide on the \emph{domain} over which you do that integral...

...the idea of model \emph{cost}...

...posterior probability might be a good objective in the (very rare)
situation that the \emph{only} audience or use is scientific, and the
priors are extremely informative...

\section{Long-term future discounted free cash flow}

The question remains: How do you go about writing down, in a
quantitative way---or calculating---your utility?  Interestingly, the
\emph{calculating} of your utility is going to turn out to be
impossible, for very deep reasons, so we will always have to live with
brutal approximations.  Let's start with an easier question: What
\emph{is} your utility?

It turns out this question \emph{has} been
answered in the realm of \emph{business}\note{I first learned about
  the LTFDFCF during a short stint as a consultant at the US internet
  company \project{Google}, where it was explicitly stated as the
  utility.  Maximization of the LTFDFCF was stated as the prime
  objective of the company.  (Now that I write these words, their
  specificity makes me wonder if I am somehow misremembering this.)},
not science (and definitely not statistics):
The objective of any entity---person or organization---is to maximize
its \emph{Long-Term Future Discounted Free Cash Flow} (LTFDFCF).
Surprised?  The thing you should \emph{really} be surprised about is
that its generality---the applicability of the LTFDFCF to science and
hypothesis testing in intellectual areas---has not been widely
recognized.  Here we break apart this concept, explain it, and argue
for it.

\paragraph{Long-term future:}
If we are rational, we care not just about what happens \emph{right
now} but about what kinds of gains (and losses) we will make in the
future.  Most of us are willing to take short-term losses for larger
long-term gains.  The present is key, but we are working towards a
future.  What future?  The next five minutes or the next thousand
years?  It turns out that the answer to this question depends on the
\emph{confidence of our predictions}...

...In scientific terms, this means...

\paragraph{Discounted:}
Okay, so we have to consider outcomes into the long-term future.  That
doesn't mean that we ignore the present; indeed, there is a full
``integral'' of the utility into the future; that integral is kept
finite by including a \emph{discount rate}.  The basic principle is
that benefit that accrues in two months is more valuable than an equal
benefit (in straight dollar terms) that accrues in twenty years...

...Subjectivity of the discount rate...

...In scientific terms, this means...

\paragraph{Free cash flow:}
Heuristically, \emph{free cash flow} is revenue less the expenses
incurred \emph{in order to generate that revenue}.  The expenses here
include short-term expenditures and amortization of \emph{past}
long-term capital expenses on equipment or facilities that were used
to generate the revenue.  This might sound a lot like \emph{profit}
but it differs in an important way: Profit is revenue less all
expenses, including \emph{present} long-term capital expenses that are
being used to create new opportunities for the future.  Profit is
always less than free cash flow, because more expenses are being
counted against profit than against free cash flow; indeed there are
companies with no profit but plenty of free cash flow; these are
generally companies that are building large new facilities or
capabilities.  You don't want to count these present-day long-term
capital expenditures against your utility, because the capapbility of
making these long-term investments is precisely \emph{why} we are
doing what we do.

In human terms, you might run a lemonade stand on a street corner to
raise money so you can start a juice bar downtown.  If you are
successful, you will use the cash flow from the lemonade stand to
build the juice bar.  In assessing the success of your lemonade stand,
you want to compare the lemonade revenue with the lemonade expenses;
the fact that you are spending every penny of that cash flow building
a new business shouldn't ``count against'' the success of the lemonade
stand.  Hence the goal of the lemonade stand was to produce free cash
flow, not profit \emph{per se}.\note{I think somehow profit is for
  \emph{investors} whereas cash flow is for \emph{owners}.  Not that
  there is obviously a clear distinction between these.}

In scientific terms, the idea is that you want to get the most benefit
from your project or your inference, given the cost \emph{of that
  project}.  That's the scientific equivalent of free cash flow.

\section{One simple example}

...Concrete example: \project{Astrometry.net}

\section{Discussion and objections}

This \documentname\ makes only one point: That probabilistic inference
does not tell you what to \emph{do}, it only puts probabilities on
parameters, hypotheses, models, and future possible data.  If you want
to know what to \emph{do}, you have to marginalize over outcomes times
some kind of utility, and that utility ought to be a long-term future
discounted free cash flow (LTFDFCF), specified in units convertible to
currency.  Apparently this is shocking to many!  Indeed I get many
objections, most of which fall into one of a few camps:

\paragraph{Objection 1: This is so subjective!}
Yes, it is subjective.  But so is all of probabilistic reasoning.  You
are making subjective judgements when you decide what data to use,
what models to consider, how to model the noise, what approximations
to make, and (as the frequentists remind us\note{Hard-core
  frequentists---with whom I have no serious disagreements (it is a
  hard way to live, but a noble way to live, as I discuss [IN WHAT
    DOCUMENT])---complain that all Bayesian methods must be wrong
  because they involve the subjective step of multiplying by a prior.
  The frequentists are not wrong: The prior is \emph{absolutely}
  subjective.  But so are the up-front model choices and
  approximations made in every inference by everyone, so in my view
  the prior pdf is, if not the least subjective part, certainly not
  the only.} when you set your prior pdfs.  All of these steps in an
inference require personal judgement.  They can be analyzed and
debated but ultimately these decisions \emph{already involve utility},
even before we get to the decision-making part of the problem.

\paragraph{Objection 2: \emph{Money?} What about happiness or love?}
Why are you so crass as to express your utility in terms of money (or
time or oil any other commodity) rather than something on a higher
plane, such as happiness or love?  The short answer is that the
fundamental assumption necessary for the LTFDFCF to be the correct
objective, is \emph{not} that money can buy happiness or that
happiness or love is convertible to money or other exchangeable
commodities.  The fundamental assumption is the much more limited
assumption that there is \emph{marginal fungibility}.

Say you are working late and you need to get home.  You are just about
to leave when a colleague says ``Can you stay 15 minutes longer?''
You say ``No, I want to get home to see my family.''  Your colleague
says ``If you stay for 15 minutes longer, I will pay you $X$.''  If
you can identify---even approximately---the money value $X$ for which
you would stay the extra 15~minutes, then you can measure a
translation between happiness (or duty or love) and LTFDFCF.  Aside
from the scenario being a bit creepy, this is the only assumption we
need to make to work in currency units, or to make marginal
translations between happiness, love, and LTFDFCF: It only needs to be
the case that you would make a small change to your behavior for an
identifiable amount of cash.

The required fungibility is \emph{marginal} simply because in any
individual decision or inference, the expected utilities at stake are
all small (relative to, say, the expected LTFDFCF you will generate in
your entire lifetime).  If you get into situations of massive expected
utilities (life-changing changes like marriage and having kids and
things like that), marginal transfer of LTFDFCF can start to break
down as a useful or useable model.

\paragraph{Objection 3: No-one knows her or his utility!}
This is absolutely true and fair!  Indeed, the word ``long-term'' in
the LTFDFCF is itself an expression (as I mention above) that the
utility is impossible to compute precisely (or accurately).

\clearpage
\markright{Notes}\theendnotes

\clearpage
\begin{thebibliography}{}\markright{References}\raggedright
\bibitem[Dark Energy Task Force(2006)]{detf}
  Dark Energy Task Force, 2006,
  ``Report of the Dark Energy Task Force'', arXiv:astro-ph/0609591
\bibitem[Hogg \etal(2010a)]{straightline}
  Hogg,~D.~W., Bovy,~J., \& Lang,~D., 2010a,
  ``Data analysis recipes:\ Fitting a model to data'', arXiv:1008.4686
\bibitem[Jaynes(2003)]{jaynes}
  Jaynes,~E.~T., 2003,
  \textit{Probability theory: the logic of science} (Cambridge University Press)
\end{thebibliography}

\end{document}
