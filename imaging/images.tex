% this file is part of the Data Analysis Recipes project
% Copyright 2012 David W. Hogg

\documentclass[12pt,twoside,pdftex]{article}
\usepackage{amssymb,amsmath,mathrsfs,../hogg_endnotes,natbib}
%%Figure caption
\makeatletter
\newsavebox{\tempbox}
\newcommand{\@makefigcaption}[2]{%
\vspace{10pt}{#1.--- #2\par}}%
\renewcommand{\figure}{\let\@makecaption\@makefigcaption\@float{figure}}
\makeatother

\newcommand{\exampleplot}[1]{%
\begin{center}%
\includegraphics[width=0.5\textwidth]{#1}%
\end{center}%
}
\newcommand{\exampleplottwo}[2]{%
\begin{center}%
\includegraphics[width=0.5\textwidth]{#1}%
\includegraphics[width=0.5\textwidth]{#2}%
\end{center}%
}

\setlength{\emergencystretch}{2em}%No overflow

\newcommand{\notenglish}[1]{\textsl{#1}}
\newcommand{\aposteriori}{\notenglish{a~posteriori}}
\newcommand{\apriori}{\notenglish{a~priori}}
\newcommand{\adhoc}{\notenglish{ad~hoc}}
\newcommand{\etal}{\notenglish{et al.}}
\newcommand{\eg}{\notenglish{e.g.}}

\newcommand{\documentname}{document}
\newcommand{\sectionname}{Section}
\newcommand{\equationname}{equation}
\newcommand{\figurenames}{\figurename s}
\newcommand{\problemname}{Exercise}
\newcommand{\problemnames}{\problemname s}
\newcommand{\solutionname}{Solution}
\newcommand{\notename}{note}

\newcommand{\note}[1]{\endnote{#1}}
\def\enotesize{\normalsize}
\renewcommand{\thefootnote}{\fnsymbol{footnote}} % the ONE footnote needs this

\newcounter{problem}
\newenvironment{problem}{\paragraph{\problemname~\theproblem:}\refstepcounter{problem}}{}
\newcommand{\affil}[1]{{\footnotesize\textsl{#1}}}

% matrix stuff
\newcommand{\mmatrix}[1]{\boldsymbol{#1}}
\newcommand{\inverse}[1]{{#1}^{-1}}
\newcommand{\transpose}[1]{{#1}^{\scriptscriptstyle \top}}
\newcommand{\mA}{\mmatrix{A}}
\newcommand{\mAT}{\transpose{\mA}}
\newcommand{\mC}{\mmatrix{C}}
\newcommand{\mCinv}{\inverse{\mC}}
\newcommand{\mQ}{\mmatrix{Q}}
\newcommand{\mS}{\mmatrix{S}}
\newcommand{\mX}{\mmatrix{X}}
\newcommand{\mY}{\mmatrix{Y}}
\newcommand{\mYT}{\transpose{\mY}}
\newcommand{\mZ}{\mmatrix{Z}}
\newcommand{\vhat}{\mmatrix{\hat{v}}}

% parameter vectors
\newcommand{\parametervector}[1]{\mmatrix{#1}}
\newcommand{\pvtheta}{\parametervector{\theta}}

% set stuff
\newcommand{\setofall}[3]{\{{#1}\}_{{#2}}^{{#3}}}
\newcommand{\allq}{\setofall{q_i}{i=1}{N}}
\newcommand{\allx}{\setofall{x_i}{i=1}{N}}
\newcommand{\ally}{\setofall{y_i}{i=1}{N}}
\newcommand{\allxy}{\setofall{x_i,y_i}{i=1}{N}}
\newcommand{\allsigmay}{\setofall{\sigma_{yi}^2}{i=1}{N}}
\newcommand{\allS}{\setofall{\mS_i}{i=1}{N}}

% other random multiply used math symbols
\renewcommand{\d}{\mathrm{d}}
\newcommand{\mean}[1]{\left<{#1}\right>}
\newcommand{\like}{\mathscr{L}}


% header stuff
\renewcommand{\MakeUppercase}[1]{#1}
\pagestyle{myheadings}
\renewcommand{\sectionmark}[1]{\markright{\thesection.~#1}}
\markboth{Probability calculus for inference}{Introduction}

\begin{document}
\thispagestyle{plain}\raggedbottom
\section*{Data analysis recipes:\\
  Making measurements with images\footnotemark}

\footnotetext{%
  The \notename s begin on page~\pageref{note:first}, including the
  license\note{\label{note:first}%
    Copyright 2012 David W. Hogg (NYU).  Right now this document
    (though publicly available) is NOT FOR DISTRIBUTION.  Eventually,
    and once it is ready, I will release it with the license ``You may
    copy and distribute this document provided that you make no
    changes to it whatsoever''.}
  and the acknowledgements\note{%
    It is a pleasure to thank
      Jo Bovy (IAS),
      Dan Foreman-Mackey (NYU),
      Hans-Walter Rix (MPIA),
      and Dan Weisz (UW)
    for discussions and comments.  This
    research was partially supported by the US National Aeronautics
    and Space Administration and National Science Foundation.}}

\noindent
David~W.~Hogg\\
\affil{Center~for~Cosmology~and~Particle~Physics, Department~of~Physics, New York University}\\
and~\affil{Max-Planck-Institut f\"ur Astronomie, Heidelberg}
%% \\[1ex]
%% A. Nother Author
%% \affil{Oxford University}

\begin{abstract}
Like, so whatever.
\end{abstract}

... Images are intensity maps, not grids of flux buckets.  The
intensity field is a quantum wave function, which is collapsed at the
detector surface, resulting in photon--electron conversions and
thereby detections.

... The ``flux bucket'' image leads to abominations like
\textsl{Drizzle}, where photons in an image are ``poured'' into
``pixel buckets'' in a different coordinate grid.  Band-limited
methods for interpolation perform the same task without loss of
resolution but they are only easy to see in the intensity map
formalism.

... Intensity is a conserved quantity; there are many consequences of
this, some of them obvious, some of them non-trivial but I am not sure
which is which.  You can't make the blue sky brighter with normal
transparent optics!  And etc.  Telescopes work by converting position
diversity at fixed direction into direction diversity at fixed
position.  All this happens at fixed (or actually slightly reduced)
intensity.

... Calibration of images, photometric, astrometric, PSF, noise
properties.  self-calibration.  What is meant by a world coordinate
system, a camera model, a photometric calibration, a flat-field, bias,
dark, and so-on.  A typical CCD noise model.

... Likelihood functions are noise models.

... Deconvolution is what \emph{everyone does always}.  People claim
to be ``against'' deconvolution, but every positional measurement of
every star ever made has been a deconvolution.

... What is blind and non-blind deconvolution and how are they done?  Examples?
 
... ``Coadding'' is only justified in extremely limited circumstances

... Coadding prohibits certain kinds of investigations

... A measurement of something in an image is inference under a model, or an approximation thereto

... A model of a set of images is no harder, in principle, than a model of a single image

... Many images of the same scene give you more information about the scene than any one image

... How do we capitalize on these sets of images?  When they are taken in different bands and under different PSFs?

... What is a catalog and what is its relationship to an image?

... What about interferometry?  Michelson vs 

... What about 2-d spectroscopy?

\clearpage
\markright{Notes}\theendnotes

\clearpage
\begin{thebibliography}{}\markright{References}\raggedright
\bibitem[Hogg \etal(2010a)]{straightline}
  Hogg,~D.~W., Bovy,~J., \& Lang,~D., 2010a,
  ``Data analysis recipes:\ Fitting a model to data'', arXiv:1008.4686
\end{thebibliography}

\end{document}
