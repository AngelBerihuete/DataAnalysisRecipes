% to-do
% -----
% - Hogg: write intro
% - Hogg: write discussion
% - Hogg: fill out comments and other stuff
% - Hogg: we need to say things about procedure; right now the document is almost entirely about the objective
% - Hogg: should we say something substantial---not superficial---about non-convexity; perhaps initializing trials with RANSAC-like initial conditions
% - Bovy: make table notes that explain that \sigma_{xy}=\sigma_x\,\sigma_y\,\rho_{xy} among other things?
% - Bovy: do something to the fake data that makes PCA differ from the right thing; this requires correlating the variance tensors, I think.
% - Bovy: make one or two plots and incorporate and cite them by \ref{}

% style notes
% -----------
% - careful with the words ``error'' and ``uncertainty''
% - careful with the words ``probability'' and ``frequency''
% - use () for function arguments, and [] for grouping/precedence
% - define macros; remember 1, 2, infinity

\documentclass[12pt]{article}
\usepackage{amssymb,amsmath,mathrsfs,deluxetable}

\newcommand{\notenglish}[1]{\textit{#1}}
\newcommand{\aposteriori}{\notenglish{a~posteriori}}
\newcommand{\apriori}{\notenglish{a~priori}}

\newcommand{\sectionname}{Section}
\newcommand{\equationname}{equation}
\newcommand{\problemname}{Exercise}
\newcommand{\commentsname}{Comments}

\newcounter{problem}
\newenvironment{problem}{\paragraph{\problemname~\theproblem:}\refstepcounter{problem}}{}
\newenvironment{comments}{\paragraph{\commentsname:}}{}

% matrix stuff
\newcommand{\mmatrix}[1]{\boldsymbol{#1}}
\newcommand{\inverse}[1]{{#1}^{-1}}
\newcommand{\transpose}[1]{{#1}^{\scriptscriptstyle \top}}
\newcommand{\mA}{\mmatrix{A}}
\newcommand{\mAT}{\transpose{\mA}}
\newcommand{\mC}{\mmatrix{C}}
\newcommand{\mCinv}{\inverse{\mC}}
\newcommand{\mQ}{\mmatrix{Q}}
\newcommand{\mX}{\mmatrix{X}}
\newcommand{\mY}{\mmatrix{Y}}
\newcommand{\mYT}{\transpose{\mY}}
\newcommand{\mZ}{\mmatrix{Z}}
\newcommand{\vhat}{\mmatrix{\hat{v}}}

% set stuff
\newcommand{\setofall}[3]{\{{#1}\}_{{#2}}^{{#3}}}
\newcommand{\allq}{\setofall{q_i}{i=1}{N}}
\newcommand{\ally}{\setofall{y_i}{i=1}{N}}
\newcommand{\allxy}{\setofall{x_i,y_i}{i=1}{N}}
\newcommand{\allsigmay}{\setofall{\sigma_{yi}^2}{i=1}{N}}
\newcommand{\allC}{\setofall{\mC_i}{i=1}{N}}

% other random multiply used math symbols
\renewcommand{\d}{\mathrm{d}}
\newcommand{\like}{\mathscr{L}}
\newcommand{\pgood}{p_{\mathrm{good}}}
\newcommand{\bperp}{b_{\perp}}
\newcommand{\mean}[1]{\left<{#1}\right>}
\newcommand{\meanZ}{\mean{\mZ}}

\begin{document}
\section*{Data analysis recipes:\ \\
  Fitting a straight line to data\footnote{
    Copyright 2009 David~W.~Hogg (david.hogg@nyu.edu) and Jo Bovy.
    You may copy and distribute this document
    provided that you make no changes to it whatsoever.}}

\noindent
David~W.~Hogg \textsl{and}
Jo~Bovy\\
\textsl{Center~for~Cosmology~and~Particle~Physics, Department~of~Physics,\\
New~York~University}

\begin{abstract}
  We go through all of the considerations involved in fitting a
  straight line to a set of points in a two-dimensional plane.
  Standard chi-squared fitting is only appropriate when there is a
  dimension along which the data points have negligible uncertainties;
  this condition is rarely met in practice.  In addition to
  considering cases of general, heterogeneous, and arbitrarily
  covariant two-dimensional uncertainties, we also look at situations
  in which there are bad data (large outliers), unknown uncertainties,
  and unknown but expected intrinsic scatter in the linear
  relationship being fit.  Above all we emphasize the importance of
  choosing a justified scalar objective, and recommend separating that
  decision from any decisions about the details of optimization.
\end{abstract}

It is conventional to begin any scientific document with an
introduction that explains why the subject matter is important.  Let
us break with tradition and observe that in almost all cases in which
scientists fit a straight line to their data, they are doing something
that is simultaneously \emph{wrong} and \emph{unnecessary}.  It is
wrong because circumstances in which a set of two dimensional
measurements---outputs from an observation, experiment, or
calculation---are truly drawn from a linear relationship is
exceedingly rare.  Indeed, any mildly nonlinear transformation of
coordinates and a truly linear relationship becomes curved.
Furthermore, even ifa relationship \emph{looks} linear, unless there
is a confidently held theoretical reason to believe that the data are
generated by a linear relationship, it probably isn't in detail; in
these cases fitting with a linear model can introduce substantial
systematic error.

Even if the investigator doesn't care that the fit is wrong, it is
likely to be unnecessary.  Why?  Because it is rare that, given a
complicated observation, experiment, or calculation, the important
\emph{result} of that work to be communicated forward in the
literature and seminars is the \emph{slope and intercept} of a
best-fit line!  Usually the full distribution of data is much more
rich, informative, and important than any simple metrics made by
fitting an overly simple model.

That said, it must be admitted that one of the most effective ways to
communicate scientific results is with catchy punchlines and compact,
approximate representations, even when those are unjustified and
unnecessary.  For this reason---and in situations where a linear fit
\emph{is} justifiable and essential---the problem of fitting a line to
data comes up very frequently in the life of a scientist.

It is a miracle with which we hope everyone reading this is familiar
that \emph{if} you have a set of two-dimensional points $(x,y)$ that
depart from a perfect, narrow, straight line only by the addition of
gaussian-distributed noise of known amplitudes in one direction (the
$y$ direction without loss of generality), the maximum-likelihood or
best-fit line for the points has a slope $m$ and intercept $b$ that
can be obtained justifiably by a perfectly linear matrix-algebra
operation known as ``least-square fitting''.  This miracle deserves
contemplation.  Once any of the input assumptions is violated (and
note there are many input assumptions), all bets are off, and there
are no consensus methods used by scientists across disciplines.  It is
one of the objectives of this document to suggest and promote some
possible consensus methods: We present below simple, straightforward,
comprehensible, and---above all---\emph{justifiable} methods for
fitting a straight line to data with general, non-trivial, and
uncertain properties.

Perhaps because there is no agreed-upon bible or cookbook to turn to,
or perhaps because most investigators would rather make some stuff up
that works ``good enough'' under deadline, or perhaps because many
realize, deep down, that much fitting is really unnecessary anyway,
there are some egregious procedures and associated errors and
absurdities in the literature.  We won't call out the guilty here, but
just make the point to remind everyone that when you cite, rely upon,
or transmit a best-fit slope or intercept reported in the literature,
you may be propagating more noise than signal.

This document is part polemic, part attempt at establishing standards,
and partnotes for our own future re-use.  We apologize in advance for
some astrophysics bias and failure to review basic ideas of linear
algebra, function optimization, and practical statistics.  Very good
reviews of the latter exist in abundance (cite Jaynes, MacKay, and
Press here).  Our focus is on the specific problem we so often face;
the general ideas appear in the context of some very concrete and
relatively realistic example problems.  The reader is encouraged to do
the \problemname s; many of these ask you to produce plots which can
be compared with the \figurename s.

\section{Standard practice}\label{sec:standard}

You have a set of $N>2$ points $(x_i,y_i)$, with known gaussian
uncertainties $\sigma_{yi}$ in the $y$ direction, and no uncertainty
at all (that is, perfect knowledge) in the $x$ direction.  You want to
find the function $f(x)$ of the form
\begin{equation}\label{eq:fofx}
f(x) = m\,x + b \quad ,
\end{equation}
where $m$ is the slope and $b$ is the intercept, that ``best fits''
the points.  What is meant by ``best fits'' is, of course, very
important, and in what follows we will have a lot to say about that.
For now, we describe standard practice:

Construct the matrices
\begin{equation}
\mY = \left[\begin{array}{c}
y_1 \\
y_2 \\
\cdots \\
y_N
\end{array}\right] \quad ,
\end{equation}
\begin{equation}
\mA = \left[\begin{array}{cc}
1 & x_1 \\
1 & x_2 \\
\multicolumn{2}{c}{\cdots} \\
1 & x_N
\end{array}\right] \quad ,
\end{equation}
\begin{equation}
\mC = \left[\begin{array}{cccc}
\sigma_{y1}^{2} & 0 & \cdots & 0 \\
0 & \sigma_{y2}^{2} & \cdots & 0 \\
\multicolumn{4}{c}{\cdots} \\
0 & 0 & \cdots & \sigma_{yN}^{2}
\end{array}\right] \quad ,
\end{equation}
where one might call $\mY$ a ``vector'', and the covariance matrix
$\mC$ can be generalized to the case in which there are covariances
between the different $y$-direction uncertainties.  The best-fit
values for the parameters $m$ and $b$ are just the components of a
column vector $\mX$ found by
\begin{equation}\label{eq:lsf}
\left[\begin{array}{c} $b$ \\ $m$ \end{array}\right]
 = \mX = \inverse{\left[\mAT\,\mCinv\,\mA\right]}
  \,\left[\mAT\,\mCinv\,\mY\right] \quad .
\end{equation}
This seems all very complicated, but it is actually the simplest thing
that can be written down that is linear, obeys matrix multiplication
rules, and has the right relative sensitivity to data of different
statistical significance.  It can be justified in one of several ways;
the linear algebra justification starts by noting that you want to
solve the equation
\begin{equation}
\mY = \mA\,\mX \quad ,
\end{equation}
but you can't because that equation is over-constrained.  So you
weight everything with the inverse of the covariance matrix (as you
would if you were doing, say, a weighted average), and then
left-multiply everything by $\mAT$ to reduce the dimensionality, and
then \equationname~(\ref{eq:lsf}) is the solution of that
reduced-dimensionality equation.

This methodology minimizes a quantity $\chi^2$ (``chi-squared''),
which is the total squared error, scaled by the uncertainties, or
\begin{equation}\label{eq:chisquared}
\chi^2
 = \sum_{i=1}^N \frac{\left[y_i - f(x_i)\right]^2}{\sigma_{yi}^2}
 \equiv \transpose{\left[\mY-\mA\,\mX\right]}
 \,\mCinv\,\left[\mY-\mA\,\mX\right]
 \quad ,
\end{equation}
that is, it finds the values for $m$ and $b$ that minimize $\chi^2$.
This, of course, is only one possible meaning of the phrase ``best
fit''; that issue is the subject of the rest of this document.

When the uncertainties are gaussian and their variances $\sigma_{yi}$
are correctly estimated, the matrix
$\inverse{\left[\mAT\,\mCinv\,\mA\right]}$ that appears in
\equationname~(\ref{eq:lsf}) is just the covariance matrix (gaussian
uncertainty variances on the diagonal, covariances off the diagonal)
for the parameters in $\mX$.  The justification of this will have to
wait for a discussion of the objective function, if it comes at all.

\begin{comments}
Even when the conditions of standard practice are met, it is
\emph{still} often done wrong!  It is not unusual to see the
individual data-point error estimates ignored, even when they are
known at least approximately.  It is also common for the problem to
get ``transposed'' such that the coordinates for which errors are
negligible (the independent variables) are put into the $\mY$ vector
and the coordinates for which errors are \emph{not} negligible (the
dependent variables) are put into the $\mA$ matrix.  In this latter
case, the procedure makes no sense at all really; it happens when the
investigator thinks of some quantity ``really being'' the dependent
variable, despite the fact that it has the smaller error.  In the
context of fitting, however, there is no meaning to these
``independent'' and ``dependent'' terms beyond the error properties.
In performing this standard fit, the investigator is effectively
assuming that the $x$ values have negligible uncertainties; if they do
\emph{not}, then the investigator is making a mistake.

In the above, cite the papers on the Hubble expansion from SNe.

Physicists should be studying linear algebra and computation at least
as much as they study calculus and differential equations.  Discuss.

The inverse covariance matrix is like a linear ``metric'' for the data
space.

Note that not only does this problem have a linear solution, the
solution is convex or unique.  This is remarkable, and almost no
perturbation of this problem (as we will see below) has either of
these properties, let alone both.
\end{comments}

\begin{problem}\label{prob:standard}
Using the standard linear algebra method of this \sectionname, fit the
straight line $y=m\,x+b$ to the $x$, $y$, and $\sigma_y$ values in
\tablename~\ref{table:data_allerr}.  Make a plot showing the points,
their uncertainties, and the best-fit line.  What is the standard
uncertainty variance $\sigma_m^2$ on the slope of the line?  Is there
anything you don't like about the result?
\end{problem}

\begin{problem}\label{prob:quadratic}
Generalize the method of this \sectionname\ to fit a general quadratic
(second order) relationship.  Add another column to matrix $\mA$
containing the values $x_i^2$, and another element to vector $\mX$
(call it $q$).  Then re-do \problemname~\ref{prob:standard} but
fitting for and plotting the best quadratic relationship
\begin{equation}
g(x) = q\,x^2 + m\,x + b \quad.
\end{equation}
\end{problem}

\begin{problem}
Explain in words why or how the linear algebra expression in
\equationname~(\ref{eq:lsf}) minimizes the squared error.  If it helps,
consider only the case in which all the data points have identical
uncertainties (so the matrix $\mC$ is trivial).
\end{problem}

\section{The objective function}\label{sec:objective}

A scientist's justification of \equationname~(\ref{eq:lsf}) cannot
appeal purely to abstract ideas of linear algebra, but must originate
from the scientific question at hand.  Here and in what follows, we
will advocate that the only reliable procedure is to use all one's
knowledge about the problem to construct a (preferably) justified,
(necessarily) scalar (or, really, one-dimensional), \emph{objective
  function} that represents monotonically the quality of the fit.  In
this framework, fitting anything to anything involves a scientific
question about the objective function representing ``goodness of fit''
and then a separate and subsequent engineering question about how to
\emph{find the optimum} and, possibly, the confidence interval or
posterior probability distribution around that optimum.  Note that in
the above section we did \emph{not} proceed according to these rules;
indeed the procedure was introduced prior to the objective function,
and the objective function was not justified.

One method for finding or creating a justified scalar objective is to
make a ``generative model'' for the data, or a statistical model for
how a data set similar to the one you have might have been generated.
In the case of the straight line fit in the presence of known,
gaussian uncertainties, one can create this generative model by
imagining that the data \emph{really do} come from a line of the form
$y = f(x) = m\,x+b$, and that the only reason that any data point
deviates from this perfect, straight line is that to this has been
added a small $y$-direction offset drawn from a gaussian distribution
of zero mean and known variance $\sigma_y^2$.  In this model, given an
independent position $x_i$, an uncertainty $\sigma_{yi}$, a slope $m$,
and an intercept $b$, the frequency distribution
$p(y_i|x_i,\sigma_{yi},m,b)$ for $y_i$ is
\begin{equation}
p(y_i|x_i,\sigma_{yi},m,b) = \frac{1}{\sqrt{2\,\pi\,\sigma_{yi}^2}}
 \,\exp\left(-\frac{[y_i - m\,x - b]^2}{2\,\sigma_{yi}^2}\right) \quad ,
\end{equation}
where this gives the expected frequency (in a hypothetical set of
repeated experiments) of getting value in the range $[y_i,y_i+\d y]$
per unit $\d y$.

This generative model provides us with a natural, justified, scalar
objective: We seek the line (parameters $m$ and $b$) that maximize the
likelihood of the observed data.  The likelihood $\like$ is
\begin{equation}\label{eq:like}
\like \propto \prod_{i=1}^N p(y_i|x_i,\sigma_{yi},m,b) \quad ,
\end{equation}
where we have used ``$\propto$'' not ``$=$'' to be careful about the
infinitesimal volume $(\d y)^N$ and a possible normalization constant
that comes from the overall probability of the data in the context of
the model space (to which we may return below).  Taking the logarithm,
\begin{eqnarray}\displaystyle
\ln\like
 & = & K - \sum_{i=1}^N \frac{[y_i - m\,x - b]^2}{2\,\sigma_{yi}^2} \nonumber\\
 & = & K - \frac{1}{2}\,\chi^2 \quad ,
\end{eqnarray}
where $K$ is some constant.  This shows that likelihood maximization
is identical to $\chi^2$ minimization and we have justified,
scientifically, the procedure of the previous section.

The Bayesian generalization of this is to say that
\begin{equation}
p(m,b|\ally,I) = \frac{p(\ally|m,b,I)}{p(\ally|I)}\,p(m,b|I) \quad ,
\end{equation}
where $m$ and $b$ are the model parameters, $\ally$ is a short-hand for
all the data $y_i$, $I$ is a short-hand for all the prior knowledge of
the $x_i$ and the $\sigma_{yi}$ and everything else about the problem,
$p(m,b|\ally,I)$ is the \emph{posterior} probability distribution for the
parameters given the data and the prior knowledge, the ratio is the
likelihood just computed, which has a frequency distribution for the
data on top and that same thing marginalized over all parameters on
the bottom (the denominator can be thought of as a normalization
constant), and $p(m,b|I)$ is the \emph{prior} probability distribution
for the parameters that represents all knowledge \emph{except} the
input data $\ally$.  Unless the prior $p(m,b|I)$ is pretty informative,
the posterior distribution function $p(m,b|\ally,I)$ here is going to look
very similar to the likelihood function in
\equationname~(\ref{eq:like}) above.

We have succeeded in justifying the standard method as optimizing a
justified, scalar objective function.  It is just the great good luck
of Gaussian distributions that that optimization is a pure linear
function of the data, and therefore trivial to implement.

\begin{comments}
Note the relationship to the weighted mean, the simple result of
combining measurements with gaussian uncertainties?  Why does the
objective have to be a ``scalar''?  Why did we put the $x_i$ and
$\sigma_{yi}$ into $I$ and not into the set of data, and when is that
appropriate?  Why did we talk about the ``frequency distribution'' for
the errors and not the ``probability distribution''?  When will I use
the words ``uncertainty'', ``noise'', and ``error''?  Quote
Neugebauer?
\end{comments}

\begin{problem}
Imagine a set of $N$ measurements $t_i$, with uncertainty variances
$\sigma_{ti}^2$, all of the same (unknown) quantity $T$.  Assuming the
generative model that each $t_i$ differs from $T$ by a
gaussian-distributed offset, taken from a gaussian with zero mean and
variance $\sigma_{ti}^2$, write down an experession for the log
likelihood $\ln\like$ for the data given the model parameter $T$.
Take a derivative and show that the maximum likelihood value for $T$
is the usual weighted mean.
\end{problem}

\section{Robustness to outliers}\label{sec:robust}

The standard linear fitting method is very sensitive to
\emph{outliers}, points that are substantially farther from the linear
relation than expected because of unmodeled experimental error or
unmodeled but rare sources of noise.  There are two general approaches
to this problem, which are not necessarily different.  The first is to
``soften'' the objective function at large deviation.  The second is
to find ways to objectively remove or reject ``bad'' points.  Both of
these are strongly preferable to sorting through the data by hand, for
reasons of subjectivity and irreproducibility that we need not state.

One straightforward way to soften the objective function relative to
$\chi^2$ is to lower the power to which residuals are raised.  For
example, if we model the frequency distribution
$p(y_i|x_i,\sigma_{yi},m,b)$ not with a gaussian but rather with a
biexponential
\begin{equation}
p(y_i|x_i,\sigma_{yi},m,b) = \frac{1}{2\,s_i}
 \,\exp\left(-\frac{|y_i-m\,x_i-b|}{s_i}\right) \quad ,
\end{equation}
where $s_i$ is an estimate of the mean absolute error, probably
correctly set to something like $s_i = \sigma_{yi}/\sqrt{2}$.
Optimization of the total log likelihood is equivalent to minimizing
\begin{equation}\label{eq:biexp}
X = \sum_{i=1}^N \frac{|y_i-f(x_i)|}{s_i} \quad ,
\end{equation}
where $f(x)$ is the straight line of \equationname~\ref{eq:fofx}.
This approach is rarely quantitatively justified, but it has the nice
property that it introduces no new parameters.

Another straightforward softening is to (smoothly) cut off the
contribution of a residual as it becomes large.  For example, replace
$\chi^2$ with
\begin{equation}\label{eq:soft}
\chi_Q^2 = \sum_{i=1}^N \frac{Q^2\,[y_i-f(x_i)]^2}
  {Q^2\,\sigma_{yi}^2+[y_i-f(x_i)]^2} \quad ,
\end{equation}
where $Q^2$ is the maximum amount a point can contribute to $\chi_Q^2$
(cite papers cited by Barron).  When each residual is small, its
contribution to $\chi_Q^2$ is nearly identical to its contribution to
the standard $\chi^2$, but as the residual gets substantial, the
contribution of the residual does not increase as the square.  This is
about the simplest robust method that introduces only one new
parameter ($Q$).

The alternative to these softenings is to try data rejection, in which
adds to the problem a set of $N$ binary integers $q_i$, one per data
point, each of which is unity if the $i$th data point is good, and
zero if the $i$th data point is bad (cite Press and Jaynes here).  In
addition, to construct an objective function one needs an additional
parameter $\pgood$, which is the \emph{prior} probability that any
individual data point is good.  All these $N+1$ extra parameters may
seem like crazy baggage, but their values can be \emph{inferred} and
\emph{marginalized out} so in the end, this method introduces \emph{no
  new parameters}.

In this situation, we don't know a consistent way to proceed that isn't
Bayesian (possibly for lack of trying), so let's go Bayesian.  In this
case, likelihood is
\begin{eqnarray}\displaystyle
\frac{p(\ally|m,b,\allq,\pgood,I)}{p(\ally|I)}
 &\propto& \prod_{i=1}^N \left[\exp\left(-\frac{[y_i-m\,x_i-b]^2}{2\,\sigma_{yi}^2}\right)\right]^q_i \quad ,
\end{eqnarray}
but there is an important prior probability
\begin{eqnarray}\displaystyle
p(m,b,\allq,\pgood|I)
 &=& p(m,b|I)\,p(\allq|\pgood,I)\,p(\pgood|I) \nonumber \\
 &=& p(m,b|I)\,\pgood^{q_i}\,[1-\pgood]^{[1-q_i]} \quad ,
\end{eqnarray}
where we have assumed that the prior probability distribution
$p(\pgood|I)$ for $\pgood$ is just flat on the interval $0<\pgood<1$
(the least informative possible assumption).

The posterior probability distribution is the product of the
(correctly normalized) likelihood times the prior.  If all we care
about are the parameters $(m,b)$ of the line, we have to marginalize
the posterior probability distribution over all choices for the binary
integers $\allq$ and all possible values for $\pgood$.  This
marginalization involves evaluating and marginalizing (over $\pgood$)
$2^N$ different likelihoods!  Of course because very few points are
true candidates for rejection, there are many ways to ``trim the
tree'' and do this quickly; the lazy can simply sample; the very lazy
can loop over all $2^N$ and go on vacation while it executes.

One unfortunate (though unavoidably correct) thing about any Bayesian
method is that it does not return an ``answer'' but rather it returns
a \emph{posterior probability distribution}.  Strictly, this posterior
distribution function \emph{is} your answer.  However, what scientists
are usually doing is not, strictly, inference, but rather,
decision-making.  That is, the investigator wants an answer, not a
distribution.  There are (at least) two reactions to this.  One is to
ignore the fundamental Bayesianism at the end, and choose simply the
``maximum \aposteriori'' (MAP) answer---the single value of $(m,b)$
that optimizes the fully marginalized (over $\pgood$ and $\allq$)
posterior probability distribution function.  This is the Bayesian's
analog of maximum likelihood.  The other reaction is to suck it up and
sample the posterior probability distribution and carry forward not
one answer to the problem but $M$ answers, each of which is drawn
fairly and independently from the posterior distribution function.
The latter is to be preferred because \textsl{(a)}~it shows the
uncertainties very clearly, and \textsl{(b)}~the sampling can be
carried forward to future inferences as an approximation to the
posterior distribution function, useful for propagating uncertainty,
or standing in as a \emph{prior} for some subsequent inference.

Another unfortunate (though unavoidably correct) thing about any
Bayesian method is that prior probabilities must be specified.  In any
situation of straight-line fitting where the data are good, these
priors do not matter very much to the final answer.  The way we have
written our likelihoods, the MAP result becomes very directly
analogous to the maximum likelihood answer when the (improper,
non-normalizable) prior ``flat in $m$, flat in $b$'' is adopted.
Usually the investigator \emph{does} have some basis for setting some
more informative priors, but a deep and sensible discussion of priors
is beyond the scope of this document.

\begin{comments}
No robust procedure will be linear, so everything must be iterated to
convergence.  Almost nothing robust will be convex, even, so there are
issues about local minima.

On bad data, you might not want to give all points the same prior
probability distribution function over $\pgood$.  One of the amusing
things about the posterior is that you can pick a particular data
point $I$ and marginalize over $(m,b)$ and all the $q_i$ \emph{except}
$q_I$.  This will return the marginalized posterior probability that
point $I$ is good.  This is good for embarassing colleagues in
meta-analyses (cite Press here).

What do you do about data rejection if you \emph{don't know the
  magnitudes of your uncertainties} $\allsigmay$?  Probably the only
way to proceed is to guess something about the \emph{relative}
uncertainties of the data points, for example that they are all
similar.

The standard method for removing sensitivity to outliers (in
astrophysics, anyway) is known as ``sigma-clipping''.  This is a
procedure that involves performing the fit, identifying the worst
outliers in a $\chi^2$ sense---the points that contribute more than
some threshold $Q^2$ to the $\chi^2$ sum, removing those, fitting
again, identifying again, and so on to convergence.  This procedure is
easy to implement, fast, and reliable (provided that the threshold
$Q^2$ is set high enough), but it has various problems that make it
less suitable than the methods described above.  One is that it is a
\emph{procedure} not an \emph{objective function}.  The procedure does
not necessarily optimize a justifiable objective.  A second is that
the procedure gives an answer that depends, in general, on the
starting point or initialization, and because there is there is no way
to compare different answers (there is no objective function), the
investigator can't decide which of two converged answers is
``better''.  You might think that the answer with least scatter
(smallest $\chi^2$ per data point) is better, but that will favor
solutions that involve rejecting most of the data; you might think the
answer that uses the most data is better, but that can have a very
large $\chi^2$.  These issues relate to the fact that the method does
not explicitly \emph{penalize} the rejection of data; this is another
bad consequence of not having an explicit objective function.  A third
problem is that the procedure does not necessarily converge to
anything non-trivial at all; if the threshold $Q^2$ gets very small,
there are situations in which all but two of the data points can be
rejected by the procedure.  All that said, with a large $Q^2$ and
standard, pretty good data, the sigma-clipping procedure is easy to
implement and fast; we have often used it ourselves.
\end{comments}

\begin{problem}
Using the biexponential objective function of
\equationname~(\ref{eq:biexp}), find the best-fit straight line
$y=m\,x+b$ for the $x$, $y$, and $\sigma_y$ values in
\tablename~\ref{table:data_allerr}.  You will have to use a non-linear
optimizer of some kind or you will have to take a sampling approach.
Make a plot showing the points, their uncertainties, and the best-fit
line.  How does this compare to the standard result you obtained in
\problemname~\ref{prob:standard}?  Do you like it better or worse?
\end{problem}

\begin{problem}
Using the soft chi-squared objective function of
\equationname~(\ref{eq:soft}), find the best-fit straight line
$y=m\,x+b$ for the $x$, $y$, and $\sigma_y$ values in
\tablename~\ref{table:data_allerr}.  You will have to use a non-linear
optimizer of some kind or you will have to take a sampling approach.
Alternatively, you can try an iterated-linear method in which you
iterate the standard linear method but modify the weights at each
iteration.  Make a plot showing the points, their uncertainties, and
the best-fit line.  How does this compare to the standard result you
obtained in \problemname~\ref{prob:standard}?  Do you like it better
or worse?
\end{problem}

\begin{problem}
Using the fully marginalized Bayesian data rejection method described
above, find the best-fit (the maximum \aposteriori) straight line
$y=m\,x+b$ for the $x$, $y$, and $\sigma_y$ values in
\tablename~\ref{table:data_allerr}.  Before choosing the MAP line,
marginalize over $\pgood$ and $\allq$ by (dumbly) looping over every
possible (non-trivial) assigment of $q_i$ values, and at each
assignment, integrate over all possible values of $\pgood$.  Make a
plot showing the points, their uncertainties, and the MAP line.  How
does this compare to the standard result you obtained in
\problemname~\ref{prob:standard}?  Do you like the MAP line better or
worse?  For extra credit, plot a sampling of 10 lines drawn from the
marginalized posterior distribution for $(m,b)$ (marginalized over
$\pgood$ and $\allq$) and plot the samples as a set of light grey or
transparent lines.  For extra extra credit, mark each data point on
your plot with the fully marginalized probability that the point is
good (that is, not rejected, or has $q=1$).
\end{problem}

\section{Assigning uncertainties to best-fit parameters}\label{sec:uncertainty}

In the standard linear-algebra method of $\chi^2$ minimization given
in \sectionname~\ref{sec:standard}, the uncertainties in the best-fit
parameters $(m,b)$ are given by the two-dimensional output covariance
matrix
\begin{equation}
\left[\begin{array}{cc}
\sigma_{b}^2 & \sigma_{mb} \\
\sigma_{mb} & \sigma_{m}^2
\end{array}\right] = \inverse{\left[\mAT\,\mCinv\,\mA\right]} \quad ,
\end{equation}
where the ordering is defined by the ordering in matrix $\mA$.  But
these best-fit uncertainties \emph{only} strictly hold under three
extremely strict conditions, almost none of which is met in real
practice: \textsl{(a)}~The uncertainties in the data points must truly
be gaussian, with variances correctly described by the
$\sigma_{yi}^2$, \textsl{(b)}~there must be no rejection of any data
or any departure from the exact, standard definition of $\chi^2$ given
in \equationname~(\ref{eq:chisquared}), and \textsl{(c)}~the
generative model of the data implied by the method---that is, that the
data are truly drawn from a negligible-scatter linear relationship and
subsequently had noise added, where the noise offsets were generated
by a gaussian process---must be an accurate description of the data.

These conditions are rarely met in practice.  Often the noise
estimates are rough (or missing entirely!), the uncertainties are
known not to be gaussian, the investigator has applied data rejection
or equivalent conditioning, and the relationship has intrinsic scatter
and curvature.  For these generic reasons, we much prefer empirical
estimates of the uncertainty in the best-fit parameters.

In the Bayesian scheme of \sectionname~\ref{sec:robust}, the output is
a posterior distribution for the parameters $(m,b)$.  This
distribution function is, in the sense we are using it here, an
empirical measurement of the uncertainties of the parameters.  The
uncertainty variances $(\sigma_m^2,\sigma_b^2)$ and the covariance
$\sigma_{mb}$ can be computed as second moments of this posterior
distribution function.  Computing the variances this way does involve
assumptions, but it does \emph{not} involve any assumption that the
model is a good fit to the data; that is, as the model becomes a bad
fit to the data (for example when the data points are not consistent
with being drawn from a narrow, straight line), these uncertainties
change in accordance.  That is in strong contrast to the elements of
the matrix $\inverse{\left[\mAT\,\mCinv\,\mA\right]}$, which don't
depend in any way on the quality of fit.

In any non-Bayesian scheme, or when the full posterior has been
discarded in favor of only the MAP value, there are still empirical
methods for determining the uncertainties in the best-fit parameters.
The two most common are \emph{bootstrap} and \emph{jackknife}.  The
first attempts to empirically create new data sets that are similar to
your actual data set.  The second measures your differential
sensitivity to each individual data point.

In bootstrap, you do the unlikely-sounding thing of drawing $N$ data
points randomly from the $N$ data points you have \emph{with
  replacement}.  That is, some data points get dropped, and some get
doubled, but the important thing is that you select each of the $N$
that you are going to use in each trial independently from the whole
set of $N$ you have.  You do this selection of $N$ points once for
each of $M$ bootstrap trials $j$.  For each of the $M$ trials $j$, you
get an estimate---by whatever method you are using (linear fitting,
fitting with rejection, optimizing some custom objection
function)---of the parameters $(m_j,b_j)$, where $j$ goes from $1$ to
$M$.  An estimate of your uncertainty variance on $m$ is
\begin{equation}
\sigma_m^2 = \frac{1}{M}\,\sum_{j=1}^M [m_j-m]^2 \quad ,
\end{equation}
where $m$ stands in for the best-fit $m$ using all the data.  The
uncertainty variance on $b$ is the same but with $[b_j-b]^2$ in the
sum, and the covariance $\sigma_{mb}$ is the same but with
$[m_j-m]\,[b_j-b]$.  Bootstrap creates a new parameter, the number $M$
of trials.  There is a huge literature on this, so we won't say
anything too specific, but one intuition is that once you have $M$
comparable to $N$, there probably isn't much else you can learn,
unless you got terribly unlucky with your random number generator.

In jackknife, you make your measurement $N$ times, each time
\emph{leaving out} data point $i$.  Again, it doesn't matter what
method you are using, for each leave-one-out trial $i$ you get an
estimate $(m_i,b_i)$ found by fitting with all the data \emph{except}
point $i$.  Then your best-fit slope becomes
\begin{equation}
m = \frac{1}{N}\,\sum_{i=1}^N m_i \quad ,
\end{equation}
and the uncertainty variance becomes
\begin{equation}
\sigma_m^2 = \frac{N-1}{N}\,\sum_{i=1}^N [m_i-m]^2 \quad ,
\end{equation}
with the obvious modifications to make $\sigma_b^2$ and
$\sigma_{mb}$.  The factor $[N-1]/N$ accounts, magically, for the
fact that the samples are not independent in any sense, and can only
be justified in the limit that everything is gaussian and all the
points are identical in their error properties.

\begin{comments}
Compare jackknife and bootstrap in terms of ``conservativeness''.
Note that neither makes sense if there is a large dynamic range in the
$\sigma_{yi}^2$.  Why do these techniques have the names they do?
They are extremely useful when you don't know or don't trust your
uncertainties $\allsigmay$.

If you are doing sigma-clipping or anything that looks like standard
least-square fitting but with small modifications, it might be
tempting to use the standard uncertainty estimate.  But that is
definitely wrong, because whatever encouraged you to do the
sigma-clipping or other modifications is probably sufficient reason to
disbelieve the standard uncertainty estimate.

Don't we have examples where a bad-fitting model can lead to very
tight Bayesian constraints; that is, aren't we being too nice to the
Bayesians?
\end{comments}

\begin{problem}
Compute the standard uncertainty $\sigma_m^2$ obtained for the slope
of the line found by the standard fit you did in
\problemname~\ref{prob:standard}.  Now make jackknife (20 trials) and
bootstrap estimates for the uncertainty $\sigma_m^2$.  How do the
uncertainties compare and which seems most reasonable, given the data
and uncertainties on the data?
\end{problem}

\section{Arbitrary two-dimensional uncertainties}\label{sec:twod}

Of course most real two dimensional data $\allxy$ come with
uncertainties in \emph{both} directions (both $x$ and $y$).  You might
not \emph{know} the amplitudes of these uncertainties, but it is
unlikely that the $x$ values are known to sufficiently high accuracy
that \emph{any} of the straight-line fitting methods given so far is
valid.  Recall that everything so far has assumed that the
$x$-direction uncertainties were negligible.

In general, when one makes a two dimensional measurement $(x_i,y_i)$,
that measurement comes uncertainties $(\sigma_{xi}^2,\sigma_{yi}^2)$
in both directions, and some covariance $\sigma_{xyi}$ between them.
These can be put together into a covariance tensor $\mC_i$
\begin{equation}
\mC_i = \left[\begin{array}{cc}
\sigma_{xi}^2 & \sigma_{xyi} \\ \sigma_{xyi} & \sigma_{yi}^2
\end{array}\right] \quad .
\end{equation}
If the errors are gaussian, or if all that is known about the
uncertainties is their variances, then the covariance tensor can be
used in a two-dimensional gaussian representation of the probability
of getting measurement $(x_i,y_i)$ when the ``true value'' (the value
you would have for this data point if it had been observed with
negligible error) is $(x,y)$:
\begin{equation}
p(x_i,y_i|\mC_i,x,y) = \frac{1}{2\,\pi\,\det(\mC_i)}
  \,\exp\left(-\frac{1}{2}\,\transpose{\left[\mZ_i - \mZ\right]}
  \,\inverse{\mC_i}\,\left[\mZ_i - \mZ\right]\right) \quad ,
\end{equation}
where we have implicitly made column vectors
\begin{equation}\label{eq:mZ}
\mZ = \left[\begin{array}{c} x \\ y \end{array}\right] \quad ; \quad
\mZ_i = \left[\begin{array}{c} x_i \\ y_i \end{array}\right] \quad .
\end{equation}

Now in the face of these general (though gaussian) two-dimensional
uncertainties, how do we fit a line?  Justified objective functions
will have something to do with the probability of the observations
$\allxy$ given the uncertainties $\allC$, as a function of properties
$(m,b)$ of the line.  As in \sectionname~\ref{sec:objective}, the
probability of the observations given model parameters $(m,b)$ is
proportional to the \emph{likelihood} of the parameters given the
observations.  We will now construct this likelihood and maximize it,
or else multiply it by a prior on the parameters and report the
posterior on the parameters.

Schematically, the construction of the likelihood involves specifying
the line (parameterized by $m,b$), finding the probability of each
observed data point given any true point on that line, and
marginalizing over all possible true points on the line.  In the end,
this procedure leads to \emph{projections} of the two-dimensional
uncertainty gaussians along the line (or onto the subspace that is
orthogonal to the line), and evaluation of those at the
\emph{projected errors}.  Projection is a standard linear algebra
technique, so we will use linear algebra (matrix) notation.

A slope $m$ can be described by a unit vector $\vhat$
\emph{orthogonal} to the line or linear relation:
\begin{equation}
\vhat
 = \frac{1}{\sqrt{1+m^2}}\,\left[\begin{array}{c}-m\\1\end{array}\right]
 = \left[\begin{array}{c}-\sin\theta\\\cos\theta\end{array}\right] \quad ,
\end{equation}
where we have defined the angle $\theta = \arctan m$ made between the
line and the $x$ axis.  The orthogonal displacement $\Delta_i$ of each
data point $(x_i,y_i)$ from the line is given by
\begin{equation}
\Delta_i = \transpose{\vhat}\,\mZ_i - b\,\cos\theta \quad ,
\end{equation}
where $\mZ_i$ is the column vector made from $(x_i,y_i)$ in
\equationname~(\ref{eq:mZ}).  Similarly, each data point's covariance
matrix $\mC_i$ projects down to an orthogonal variance $\Sigma_i^2$ given by
\begin{equation}\label{eq:Sigma}
\Sigma_i^2 = \transpose{\vhat}\,\mC_i\,\vhat
\end{equation}
and then the log likelihood for $(m,b)$ or $(\theta,b\,\cos\theta)$
can be written as
\begin{equation}\label{eq:twodlike}
\ln\like = K - \sum_{i=1}^N \frac{\Delta_i^2}{2\,\Sigma_{i}^2} \quad ,
\end{equation}
where $K$ is some constant.  \emph{This} likelihood can be maximized,
and the resulting values of $(m,b)$ are justifiably the best-fit
values.  The only modification we would suggest is performing the fit
or likelihood maximization not in terms of $(m,b)$ but rather
$(\theta,\bperp)$, where $\bperp$ is the perpendicular distance of the
line from the origin or $b\,\cos\theta$.  This removes the paradox
that the standard ``prior'' assumption of standard straight-line
fitting treats all \emph{slopes} $m$ equally, putting way too much
attention on angles near $\pm\pi/2$.  The Bayesian must set a prior.
Again, there are many choices here, but the most natural is
\emph{something} like flat in $\theta$ and flat in $\bperp$ (the
latter not proper).

The implicit generative model here is that there are $N$ points with
true values that lie precisely on a narrow, linear relation in the
$x$--$y$ plane.  To each of these true points a gaussian offset has
been added to make each observed point $(x_i,y_i)$, where the offset
was drawn from the two-dimensional gaussian with covariance tensor
$\mC_i$.  As usual, if this generative model is a good approximation
to the properties of the data set, the method works very well.  Of
course there are many situations in which this is \emph{not} a good
approximation.  Below, we consider the (very common) case that the
relationship is near-linear but not \emph{narrow}, so there is an
intrinsic width or scatter in the true relationship.  Another case is
that there are outliers; this can be taken care of by methods very
precisely analogous to the methods in \sectionname~\ref{sec:robust}.

\begin{comments}
It is true that standard least-squares fitting is easy and simple;
presumably this explains why it is used so often when it is
inappropriate.  We hope to have convinced some that doing something
justifiable and sensible when there are uncertainties in both
dimensions---when standard linear fitting is inappropriate---is
neither difficult nor complex.

That said, there is something fundamental wrong with the alleged
generative model of this section, and it is that the model generates
\emph{only} the orthogonal displacements of the points orthogonal to
the linear relationship.  The model is completely unspecified for the
distribution \emph{along} the relationship.  This probably means that
although this method works, there is something fishy or improper
involved.

In the astrophysics literature (see, for example, the Tully--Fisher
literature), there is a tradition, when there are errors in both
directions, of fitting the ``forward'' and ``reverse''
relations---that is, fitting $y$ as a function of $x$ and then $x$ as
a function of $y$---and then splitting the difference between the two
slopes so obtained, or treating the difference between the slopes as a
systematic uncertainty.  It \emph{is} a systematic uncertainty in the
sense that if you are doing something \emph{unjustifiable} you will
certainly introduce large systematics!  This forward--reverse
procedure is not justifiable and it gives results which differ by
substantially more than the uncertainty in the procedure given in this
\sectionname.

Another common method for finding the linear relationship in data when
there are errors in both direction is \emph{principal components
  analysis}.  The manifold reasons \emph{not} to use this here are
beyond the scope of this document, but in a nutshell: The method of
principal components analysis \emph{does} return a linear relationship
for a data set, in the form of the dominant principal component.
However, this is the dominant principal component of the observed
data, not of the underlying linear relationshiop that, when noise is
added, generates the observations.  For this reason, the output of PCA
will be strongly drawn or affected by the individual data point noise
covariances $\mC_i$.  This point is a subtle one, but in astrophysics
it is almost certainly having a large effect in the standard
literature on the ``fundamental plane'' of elliptical galaxies, and in
other areas where a fit is being made to data with substantial
uncertainties.  Of course PCA is another trivial, linear method, so it
is often useful despite its inapplicability; and it becomes applicable
in the limit that the data uncertainties are negligible (but in that
case everything is trivial, no?).
\end{comments}

\begin{problem}\label{prob:twod}
Using the method of this \sectionname, fit the straight line
$y=m\,x+b$ to the $x$, $y$, $\sigma_x^2$, $\sigma_{xy}$, and $\sigma_y^2$
values of points XX--XX taken from \tablename~\ref{table:data_allerr}.
Make a plot showing the points, their two-dimensional uncertainties
(show them as one-sigma ellipses), and the best-fit line.
\end{problem}

\begin{problem}
Repeat \problemname~\ref{prob:twod}, but using \emph{all} of the data
in \tablename~\ref{table:data_allerr}.  Some of the points are now
outliers, so your fit may look worse.  Follow the fit by a robust
procedure analogous to the Bayesian procedure with good data
probability $\pgood$ and classifications $\allq$ described in
\sectionname~\ref{sec:robust}.  Use something sensible for the prior
probability distribution for $(m,b)$, as discussed above.  Plot the
two results with the data and uncertainties.  For extra credit, plot a
sampling of 10 lines drawn from the marginalized posterior
distribution for $(m,b)$ (marginalized over $\pgood$ and $\allq$) and
plot the samples as a set of light grey or transparent lines.  For
extra extra credit, mark each data point on your plot with the fully
marginalized probability that the point is good (that is, not
rejected, or has $q=1$).
\end{problem}

\begin{problem}
Perform the forward--reverse fitting procedure on points XX--XX from
\tablename~\ref{table:data_allerr}.  That is, fit a straight line to
the $y$ values of the points XX--XX, using the $y$-direction
uncertainties $\sigma_y^2$ only, by the standard method described in
\sectionname~\ref{sec:standard}.  Now transpose the problem and fit
the same data but fitting the $x$ values using the $x$-direction
uncertainties $\sigma_x^2$ only.  Make a plot showing the data points,
the $x$-direction and $y$-direction uncertainties, and the two
best-fit lines.  Comment.
\end{problem}

\begin{problem}
Perform principal components analysis on points XX--XX from
\tablename~\ref{table:data_allerr}.  That is, diagonalize the $2\times
2$ matrix $\mQ$ given by
\begin{equation}
\mQ = \sum_{i=1}^N\,\left[\mZ_i-\meanZ\right]
  \,\transpose{\left[\mZ_i-\meanZ\right]} \quad ,
\end{equation}
\begin{equation}
\meanZ = \frac{1}{N}\,\sum_{i=1}^N\,\mZ_i
\end{equation}
Find the eigenvector of $\mQ$ with the largest eigenvalue.  Now make a
plot showing the data points, and the line that goes through the mean
$\meanZ$ of the data with the slope corresponding to the direction of
the principal eigenvector.  Comment.
\end{problem}

\section{Intrinsic scatter}\label{sec:scatter}

So far, everything we have done has implicitly assumed that there
truly is a narrow linear relationship between $x$ and $y$ (or there
would be if they were both measured with negligible uncertainties).
The words ``narrow'' and ``linear'' are both problematic, since there
are very few problems in science, especially in astrophysics, in which
a relationship between two observables is expected to be either.  To
generalize beyond linear is to go beyond the scope of this document,
so let's consider relationships that are linear but not narrow.  That
is, in general we will have to consider relationships that have finite
width.  This gets into the complicated area of estimating density
given finite and noisy samples; again this is a huge subject so we
will only consider two cases.

The first is the generalization of the method in
\sectionname~\ref{sec:twod} to permit an intrinsic gaussian variance
$V$ to the relationship, orthogonal to the line.  In this case, the
parameters of the relationship become $(\theta,\bperp,V)$.  In this
case, each data point can be treated as being drawn from a projected
distribution function that is a \emph{convolution} of the projected
uncertainty gaussian, of variance $\Sigma_i^2$ defined in
\equationname~(\ref{eq:Sigma}), with the intrinsic scatter gaussian of
variance $V$.  Convolution of gaussians is trivial and the likelihood
in this case becomes
\begin{equation}
\ln\like = K - \sum_{i=1}^N \frac{1}{2}\,\ln(\Sigma_{i}^2+V)
 - \sum_{i=1}^N \frac{\Delta_i^2}{2\,[\Sigma_{i}^2+V]} \quad ,
\end{equation}
where again $K$ is a constant, everything else is defined as it is in
\equationname~(\ref{eq:twodlike}), and an additional term has
appeared to penalize very broad variances (which, because they are
less specific than small variances, have lower likelihoods).
Actually, that term existed in \equationname~(\ref{eq:twodlike}) as
well, but because there was no $V$ parameter, it did not enter into
any optimization so we absorbed it into $K$.

As we noted in the \commentsname\ of \sectionname~\ref{sec:twod},
there are limitations to this procedure because it models only the
distribution \emph{orthogonal} to the relationship.  Probably all good
methods fully model the intrinsic two-dimensional density function,
the function that, when convolved with each data point's intrinsic
uncertainty gaussian, creates a distribution function from which that
data point is a single sample.  Such methods fall into the
intersection of density estimation and deconvolution, and completely
general methods exist (BOVY: cite Bovy here).

\begin{comments}
Why should you never subtract out the variance in quadrature?  Hogg
can think of two reasons.
\end{comments}

\begin{problem}\label{prob:intrinsic}
Re-do \problemname~\ref{prob:twod}, but now allowing for an orthogonal
intrinsic gaussian variance $V$.  Re-make the plot, showing not the
best-fit line but rather the $\pm\sqrt{V}$ lines for the
maximum-likelihood \emph{intrinsic} relation.
\end{problem}

\begin{problem}
Re-do \problemname~\ref{prob:intrinsic} but as a Bayesian, with
sensble Bayesian priors on $(\theta,\bperp,V)$.  Find and marginalize
the posterior distribution over $(\theta,\bperp)$ to generate a
marginalized posterior probability for the intrinsic variance
parameter $V$.  Plot this posterior with the 95 and 99~percent
\emph{upper limits} on $V$ marked.  Why did we ask only for upper
limits?
\end{problem}

\section{Discussion}

What did we \emph{not} discuss?

Non-gaussian errors: Difference between error, noise, and uncertainty.
What are you really assuming?  It could just be that you mis-estimated
your variance.  How do you deal, in general, with other kinds of error
distributions?

Consider either here or above some kind of foreground/background EM
modeling, a la Dustin.

How to deal with upper and lower limits?  Might need to bring in the
idea of maximum entropy here.  Not unlike the non-gaussian noise
situation.

Goodness of fit and model selecton?

\paragraph{Acknowlegments}
We owe a debt above all to Sam Roweis (Toronto), who taught us to find
and optimize justified scalar objective functions.  In addition it is
a pleasure to thank Mike Blanton (NYU) and Dustin Lang (Toronto,
Princeton) for discussions.  This research was partially supported by
NASA (ADP grant NNX08AJ48G) and a Research Fellowship of the Alexander
von Humboldt Foundation.  This research made use of the Python
programming language and the open-source Python packages scipy, numpy,
and matplotlib.

\clearpage
%Any changes to this table should be made IN THE CODE
\begin{deluxetable}{rrrrrr@{.}l}
\tablecolumns{7}
\tablehead{ID &$x$ & $y$ & $\sigma_y$ & $\sigma_x$ &  \multicolumn{2}{c}{$\rho_{xy}$}}
\tablewidth{0pt}
\startdata
1 & 201 & 592 & 61 & 9 & -0 & 84\\
2 & 244 & 401 & 25 & 4 & 0 & 31\\
3 & 47 & 583 & 38 & 11 & 0 & 64\\
4 & 287 & 402 & 15 & 7 & -0 & 27\\
5 & 203 & 495 & 21 & 5 & -0 & 33\\
6 & 58 & 173 & 15 & 9 & 0 & 67\\
7 & 210 & 479 & 27 & 4 & -0 & 02\\
8 & 202 & 504 & 14 & 4 & -0 & 05\\
9 & 198 & 510 & 30 & 11 & -0 & 84\\
10 & 158 & 416 & 16 & 7 & -0 & 69\\
11 & 165 & 393 & 14 & 5 & 0 & 30\\
12 & 201 & 442 & 25 & 5 & -0 & 46\\
13 & 157 & 317 & 52 & 5 & -0 & 03\\
14 & 131 & 311 & 16 & 6 & 0 & 50\\
15 & 166 & 400 & 34 & 6 & 0 & 73\\
16 & 160 & 337 & 31 & 5 & -0 & 52\\
17 & 186 & 423 & 42 & 9 & 0 & 90\\
18 & 125 & 334 & 26 & 8 & 0 & 40\\
19 & 218 & 533 & 16 & 6 & -0 & 78\\
20 & 146 & 344 & 22 & 5 & -0 & 56\\
\enddata
\label{table:data_allerr}
\end{deluxetable}


\end{document}
