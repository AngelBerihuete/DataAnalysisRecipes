% Copyright 2011 David W. Hogg (NYU).
% All rights reserved (for now).

% to-do
% -----
% - write
% - edit
% - post

\documentclass[12pt, letterpaper]{article}
\usepackage{amssymb,amsmath,mathrsfs}

\newcommand{\data}{D}
\newcommand{\alldata}{\{D_i\}_{{\mathrm{all}}\,i}}
\newcommand{\galpars}{\theta}
\newcommand{\hyperpars}{\alpha}
\newcommand{\amplitude}{C}
\newcommand{\dd}{\mathrm{d}}
\newcommand{\normal}{\mathscr{N}}

\begin{document}
\section*{Data analysis recipes:\\
          Classification at low signal-to-noise}

\begin{abstract}
In classification problems of interest to physicists and astronomers,
it is not usually appropriate to use ``black-box'' classifiers (SVMs,
neural nets, and the like) trained on well-classified training sets.
This is because training data tend to be different---in selection,
calibration, signal-to-noise, or other important properties---from the
data on which the classification is to be performed.  Furthermore, the
data set to be classified is usually itself the best source of data
about the classification; if it weren't, the investigator wouldn't be
wanting to classify it!  Furthermore, the investigator usually wants
(whether or not he or she knows it) probabilistic classifications, for
use in subsequent utility-based hard classification or
forward-modeling projects.  Here we show that if the investigator has
generally useful information about the different classes and a noise
model for the data, hierarchical probabilistic inference can be used
to construct extremely successful classifiers even with no training
data whatsoever.  The method works by building a mixture-of-classes
model for the observed data, finding class probabilities and parameter
distributions that optimize the probability of the data under the
model (the marginalized likelihood), and then competing the classes
(mixture components) probabilistically.
\end{abstract}

The general problem of classification with noisy data has generated an
enormous literature, and a wide range of very capable and useful
techniques.  Most of these techniques are useless for the kind of
problems and objectives that physical scientists encounter: These are
problems where there are not extremely reliable training sets, where
the noise in the data is at least partially understood, and where the
objective is to propagate uncertainty through the classification
process.

For example, consider the astronomical problem of star--galaxy
separation: The goal is to take a set of images of the sky through
different filter bandpasses, and determine which of the astronomical
sources in the image are stars and which are galaxies.

Equations we will encounter include the following:
\begin{eqnarray}\displaystyle
p(\data_i|S,I) &=& \normal(\data_i|0,\sigma_i^2) \\
p(\data_i|G,\galpars,I) &=& \normal(\data_i|\galpars_i,\sigma_i^2) \\
p(\data_i|G,I,\hyperpars) &=& \int p(\data_i|G,\galpars,I)\,p(\galpars|G,\hyperpars)\,\dd \galpars \\
p(\data_i|I,\hyperpars) &=& \sum_{q=S}^G p(\data_i|q,I,\hyperpars)\,P_q \\
1 &=& \sum_{q=S}^G P_q \\
p(\alldata|I,\hyperpars) &=& \prod_i p(\data_i|I,\hyperpars) \\
p(\data_i|k,\amplitude_{ki},I) &=& \normal(\data_i|\amplitude_{ki},\Sigma_i^2) \\
p(\data_i|k,I,\hyperpars) &=& \int p(\data_i|k,\amplitude_{ki},I)\,p(\amplitude_{ki}|k,\hyperpars)\,\dd \amplitude_{ki} \\
p(\data_i|I,\hyperpars) &=& \sum_k p(\data_i|k,I,\hyperpars)\,P_k \\
1 &=& \sum_k P_k \\
p(\alldata|I,\hyperpars) &=& \prod_i p(\data_i|I,\hyperpars) \\
\end{eqnarray}

\end{document}
