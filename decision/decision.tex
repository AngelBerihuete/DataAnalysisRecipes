% this file is part of the Data Analysis Recipes project
% Copyright 2012 David W. Hogg

\documentclass[12pt,twoside,pdftex]{article}
%%Figure caption
\makeatletter
\newsavebox{\tempbox}
\newcommand{\@makefigcaption}[2]{%
\vspace{10pt}{#1.--- #2\par}}%
\renewcommand{\figure}{\let\@makecaption\@makefigcaption\@float{figure}}
\makeatother

\newcommand{\exampleplot}[1]{%
\begin{center}%
\includegraphics[width=0.5\textwidth]{#1}%
\end{center}%
}
\newcommand{\exampleplottwo}[2]{%
\begin{center}%
\includegraphics[width=0.5\textwidth]{#1}%
\includegraphics[width=0.5\textwidth]{#2}%
\end{center}%
}

\setlength{\emergencystretch}{2em}%No overflow

\newcommand{\notenglish}[1]{\textsl{#1}}
\newcommand{\aposteriori}{\notenglish{a~posteriori}}
\newcommand{\apriori}{\notenglish{a~priori}}
\newcommand{\adhoc}{\notenglish{ad~hoc}}
\newcommand{\etal}{\notenglish{et al.}}
\newcommand{\eg}{\notenglish{e.g.}}

\newcommand{\documentname}{document}
\newcommand{\sectionname}{Section}
\newcommand{\equationname}{equation}
\newcommand{\figurenames}{\figurename s}
\newcommand{\problemname}{Exercise}
\newcommand{\problemnames}{\problemname s}
\newcommand{\solutionname}{Solution}
\newcommand{\notename}{note}

\newcommand{\note}[1]{\endnote{#1}}
\def\enotesize{\normalsize}
\renewcommand{\thefootnote}{\fnsymbol{footnote}} % the ONE footnote needs this

\newcounter{problem}
\newenvironment{problem}{\paragraph{\problemname~\theproblem:}\refstepcounter{problem}}{}
\newcommand{\affil}[1]{{\footnotesize\textsl{#1}}}

% matrix stuff
\newcommand{\mmatrix}[1]{\boldsymbol{#1}}
\newcommand{\inverse}[1]{{#1}^{-1}}
\newcommand{\transpose}[1]{{#1}^{\scriptscriptstyle \top}}
\newcommand{\mA}{\mmatrix{A}}
\newcommand{\mAT}{\transpose{\mA}}
\newcommand{\mC}{\mmatrix{C}}
\newcommand{\mCinv}{\inverse{\mC}}
\newcommand{\mQ}{\mmatrix{Q}}
\newcommand{\mS}{\mmatrix{S}}
\newcommand{\mX}{\mmatrix{X}}
\newcommand{\mY}{\mmatrix{Y}}
\newcommand{\mYT}{\transpose{\mY}}
\newcommand{\mZ}{\mmatrix{Z}}
\newcommand{\vhat}{\mmatrix{\hat{v}}}

% parameter vectors
\newcommand{\parametervector}[1]{\mmatrix{#1}}
\newcommand{\pvtheta}{\parametervector{\theta}}

% set stuff
\newcommand{\setofall}[3]{\{{#1}\}_{{#2}}^{{#3}}}
\newcommand{\allq}{\setofall{q_i}{i=1}{N}}
\newcommand{\allx}{\setofall{x_i}{i=1}{N}}
\newcommand{\ally}{\setofall{y_i}{i=1}{N}}
\newcommand{\allxy}{\setofall{x_i,y_i}{i=1}{N}}
\newcommand{\allsigmay}{\setofall{\sigma_{yi}^2}{i=1}{N}}
\newcommand{\allS}{\setofall{\mS_i}{i=1}{N}}

% other random multiply used math symbols
\renewcommand{\d}{\mathrm{d}}
\newcommand{\mean}[1]{\left<{#1}\right>}
\newcommand{\like}{\mathscr{L}}


% header stuff
\renewcommand{\MakeUppercase}[1]{#1}
\pagestyle{myheadings}
\renewcommand{\sectionmark}[1]{\markright{\thesection.~#1}}
\markboth{Extracting information from images}{Introduction}

\begin{document}
\thispagestyle{plain}\raggedbottom
\section*{Data analysis recipes:\\
  Making decisions}

\footnotetext{%
  The \notename s begin on page~\pageref{note:first}, including the
  license\note{\label{note:first}%
    Copyright 2012 David W. Hogg (NYU).  Right now this document
    (though publicly available) is NOT FOR DISTRIBUTION.  Eventually,
    and once it is ready, I will release it with the license ``You may
    copy and distribute this document provided that you make no
    changes to it whatsoever''.}
  and the acknowledgements\note{%
    It is a pleasure to thank
      Jo Bovy (IAS),
      Dan Foreman-Mackey (NYU),
      Dustin Lang (CMU), and
      Sam Roweis (deceased)
    for discussions and comments.  This
    research was partially supported by the US National Aeronautics
    and Space Administration and National Science Foundation.}}

\noindent
David~W.~Hogg\\
\affil{Center~for~Cosmology~and~Particle~Physics, Department~of~Physics, New York University}\\
and~\affil{Max-Planck-Institut f\"ur Astronomie, Heidelberg}
%% \\[1ex]
%% A. Nother Author
%% \affil{Oxford University}

\begin{abstract}
Probabilistic inference at its best gives probabilistic information
about parameters and models.  In many cases of importance---as with
experimental design or purchasing or delivery of results---it is
necessary to make a hard decision or choice.  It is almost never the
right move to choose the most probable model or option; it is rare
that the most probable is the option that delivers the most benefit to
the decider.  Instead the decider ought to specify a utility function
and work to optimize expected utility, or minimize expected loss, or
minimize maximal loss.  In real-world situations of building
experiments or operating an organization, to be generally useful the
utility must be specified in (the equivalent of) currency (dollar)
units.  Whether the decision is about which observation to make next,
which of several hypotheses to reject, or which of very different
research programs to pursue, the employed utility function should be
an approximation to the decider's \emph{long-term future discounted
  free cash flow}.
\end{abstract}

Back in 2005-ish, the cosmology community underwent a soul-searching
exercise about projects that are intended to investigate the nature of
\emph{dark energy}, or the mechanical cause of the late-time
acceleration of the Universe's acceleration.  The Dark-Energy Task
Force (DETF) ended up recommending that projects be ranked on the
basis of their ability to constrain two dark-energy parameters, $w_0$
and $w_a$.  In particular, they argued that the relevant scalar or
utility or figure of merit (FoM) to consider is the inverse of the
area inside the 95-percent confidence interval (implicitly a
likelihood contour) in the $w_0$--$w_a$ plane.\note{\cite{detf}.}

In many ways this was a breakthrough: A committee recommends a
quantitative method for comparing (necessarily expensive, complex, and
lengthy) projects, and describes that quantitative method in terms of
merit or utility.  Furthermore, the proposed FoM scalar relates to the
informaton content in the data delivered by the experiments, so it is
eminently sensible.

However, there is an equally valid point of view in which this
recommendation was \emph{useless}.  The reason is that although the
FoM permits the community to compare projects in terms of information
delivered about dark energy, no two projects deliver the same FoM, no
two projects have the same total cost, and no two projects deliver
results at the same point in the future.  If one project delivers an
FoM of 1200, costs $8\times 10^6$~USD, and takes three years, and
another delivers an FoM of 1950 and costs $17\times 10^6$~USD, and
takes 2.5 years, how are we to compare them?  We need to be able to
understand the trade-offs between money, FoM, and time.  And sure
enough, although the DETF was able to classify projects into
categories (FoM bins), it was not able to rank the relative value of
very qualitatively different projects.

\section{What is a decision?}

\section{Utility}

minimax and utility maximization

\section{Long-term future discounted free cash flow}

Long-term future.

Discounted.

Free cash flow.

Assumptions under which this is possible.

\clearpage
\markright{Notes}\theendnotes

\clearpage
\begin{thebibliography}{}\markright{References}\raggedright
\bibitem[Dark Energy Task Force(2006)]{detf}
  Dark Energy Task Force, 2006,
  ``Report of the Dark Energy Task Force'', arXiv:astro-ph/0609591
\bibitem[Hogg \etal(2010a)]{straightline}
  Hogg,~D.~W., Bovy,~J., \& Lang,~D., 2010a,
  ``Data analysis recipes:\ Fitting a model to data'', arXiv:1008.4686
\end{thebibliography}

\end{document}
