\documentclass[12pt,twoside]{article}
\usepackage{amssymb,amsmath,mathrsfs,deluxetable}
\usepackage{natbib}
\usepackage{float,graphicx}
% hypertex insanity
\usepackage{color,hyperref}
\definecolor{linkcolor}{rgb}{0,0,0.25}
\hypersetup{
  colorlinks=true,        % false: boxed links; true: colored links
  linkcolor=linkcolor,    % color of internal links
  citecolor=linkcolor,    % color of links to bibliography
  filecolor=linkcolor,    % color of file links
  urlcolor=linkcolor      % color of external links
}
%%Figure caption
\makeatletter
\newsavebox{\tempbox}
\newcommand{\@makefigcaption}[2]{%
\vspace{10pt}{#1.--- #2\par}}%
\renewcommand{\figure}{\let\@makecaption\@makefigcaption\@float{figure}}
\makeatother

\setlength{\emergencystretch}{2em}%No overflow

\newcommand{\notenglish}[1]{\textsl{#1}}
\newcommand{\aposteriori}{\notenglish{a~posteriori}}
\newcommand{\apriori}{\notenglish{a~priori}}
\newcommand{\adhoc}{\notenglish{ad~hoc}}
\newcommand{\etal}{\notenglish{et al.}}
\newcommand{\eg}{\notenglish{e.g.}}

\newcommand{\documentname}{document}
\newcommand{\sectionname}{Section}
\newcommand{\equationname}{equation}
\newcommand{\problemname}{Exercise}
\newcommand{\solutionname}{Solution}
\newcommand{\commentsname}{Comments}

\newcounter{problem}
\newenvironment{problem}{\paragraph{\problemname~\theproblem:}\refstepcounter{problem}}{}
\newenvironment{comments}{\paragraph{\commentsname:}}{}

% matrix stuff
\newcommand{\mmatrix}[1]{\boldsymbol{#1}}
\newcommand{\inverse}[1]{{#1}^{-1}}
\newcommand{\transpose}[1]{{#1}^{\scriptscriptstyle \top}}
\newcommand{\mA}{\mmatrix{A}}
\newcommand{\mAT}{\transpose{\mA}}
\newcommand{\mC}{\mmatrix{C}}
\newcommand{\mCinv}{\inverse{\mC}}
\newcommand{\mE}{\mmatrix{E}}
\newcommand{\mQ}{\mmatrix{Q}}
\newcommand{\mS}{\mmatrix{S}}
\newcommand{\mX}{\mmatrix{X}}
\newcommand{\mY}{\mmatrix{Y}}
\newcommand{\mYT}{\transpose{\mY}}
\newcommand{\mZ}{\mmatrix{Z}}
\newcommand{\vhat}{\mmatrix{\hat{v}}}

% parameter vectors
\newcommand{\parametervector}[1]{\mmatrix{\vec{#1}}}
\newcommand{\pvtheta}{\parametervector{\theta}}

% set stuff
\newcommand{\setofall}[3]{\{{#1}\}_{{#2}}^{{#3}}}
\newcommand{\allq}{\setofall{q_i}{i=1}{N}}
\newcommand{\allx}{\setofall{x_i}{i=1}{N}}
\newcommand{\ally}{\setofall{y_i}{i=1}{N}}
\newcommand{\allxy}{\setofall{x_i,y_i}{i=1}{N}}
\newcommand{\allsigmay}{\setofall{\sigma_{yi}^2}{i=1}{N}}
\newcommand{\allC}{\setofall{\mC_i}{i=1}{N}}

% other random multiply used math symbols
\renewcommand{\d}{\mathrm{d}}
\newcommand{\like}{\mathscr{L}}
\newcommand{\Pbad}{P_{\mathrm{b}}}
\newcommand{\Ybad}{Y_{\mathrm{b}}}
\newcommand{\Vbad}{V_{\mathrm{b}}}
\newcommand{\bperp}{b_{\perp}}
\newcommand{\mean}[1]{\left<{#1}\right>}
\newcommand{\meanZ}{\mean{\mZ}}
\newcommand{\best}{\mathrm{best}}

% header stuff
\renewcommand{\MakeUppercase}[1]{#1}
\pagestyle{myheadings}
\renewcommand{\sectionmark}[1]{\markright{\thesection.~#1}}
\markboth{Fitting a straight line to data}{}

\begin{document}
\thispagestyle{plain}\raggedbottom
%% \section*{Data analysis recipes:\ \\
%%   Comparing models of different complexity\footnote{
%%     Copyright 2010 by the authors.
%%     \textbf{This is a DRAFT version dated 2010-03-28.
%%     Please do not distribute this document.}}}
%% %    You may copy and distribute this document
%% %    provided that you make no changes to it whatsoever.}}

%% \noindent
%% David~W.~Hogg\footnote{\textsl{Center~for~Cosmology~and~Particle~Physics, Department~of~Physics, New York University} and \textsl{Max-Planck-Institut f\"ur Astronomie, Heidelberg}},
%% Jo~Bovy\footnote{\textsl{Center~for~Cosmology~and~Particle~Physics, Department~of~Physics, New York University}}, \&
%% Dustin~Lang\footnote{\textsl{Department of Computer Science, University of Toronto} and \textsl{Princeton University Observatory}}

%% \begin{abstract}
%%   TBD
%% \end{abstract}

\begin{equation}
y_i = a_{ij}\,x_j + e_i
  \quad ,
\end{equation}
\begin{equation}
\mY = \mA\,\mX + \mE
  \quad ,
\end{equation}
\begin{equation}
\chi^2 = \sum_{i=1}^N
  \frac{\left[y_i - a_{ij}\,x_j\right]^2}{\sigma_{yi}^2}
  \quad ,
\end{equation}
\begin{equation}
\chi^2 = \transpose{\left[\mY-\mA\,\mX\right]}
  \,\mCinv\,\left[\mY-\mA\,\mX\right]
  \quad ,
\end{equation}
\begin{equation}
\mX_\best = \inverse{\left[\mAT\,\mCinv\,\mA\right]}
  \,\left[\mAT\,\mCinv\,\mY\right]
  \quad ,
\end{equation}
\begin{equation}
\chi_r^2 = \sum_{i=1}^N
  \frac{\left[y_i - a_{ij}\,x_j\right]^2}{\sigma_{yi}^2}
  + \epsilon\,\sum_{j=1}^{K-1}\left[x_{j+1}-x_j\right]^2
  \quad ,
\end{equation}
\begin{equation}
\chi_r^2 = 
\end{equation}
\begin{equation}
\mX_\best = \inverse{\left[\mAT\,\mCinv\,\mA + \epsilon\,\mQ\right]}
  \,\left[\mAT\,\mCinv\,\mY\right]
  \quad ,
\end{equation}
\begin{equation}
\mQ = \left[\begin{array}{cccccc}
    1 &-1 & 0 & 0 & 0 & 0 \\
   -1 & 2 &-1 & 0 & 0 & 0 \\
    0 &-1 & 2 &-1 & 0 & 0 \\
    0 & 0 &-1 & 2 &-1 & 0 \\
    0 & 0 & 0 &-1 & 2 &-1 \\
    0 & 0 & 0 & 0 &-1 & 1 \\
  \end{array}\right]
\end{equation}

\end{document}
