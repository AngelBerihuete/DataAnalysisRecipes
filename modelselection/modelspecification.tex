% This file is part of the Data Analysis Recipes project.
% Copyright 2011, 2012, 2013 David W. Hogg (NYU).

% to-do
% -----
% - Write!

\documentclass[12pt,twoside]{article}
%%Figure caption
\makeatletter
\newsavebox{\tempbox}
\newcommand{\@makefigcaption}[2]{%
\vspace{10pt}{#1.--- #2\par}}%
\renewcommand{\figure}{\let\@makecaption\@makefigcaption\@float{figure}}
\makeatother

\newcommand{\exampleplot}[1]{%
\begin{center}%
\includegraphics[width=0.5\textwidth]{#1}%
\end{center}%
}
\newcommand{\exampleplottwo}[2]{%
\begin{center}%
\includegraphics[width=0.5\textwidth]{#1}%
\includegraphics[width=0.5\textwidth]{#2}%
\end{center}%
}

\setlength{\emergencystretch}{2em}%No overflow

\newcommand{\notenglish}[1]{\textsl{#1}}
\newcommand{\aposteriori}{\notenglish{a~posteriori}}
\newcommand{\apriori}{\notenglish{a~priori}}
\newcommand{\adhoc}{\notenglish{ad~hoc}}
\newcommand{\etal}{\notenglish{et al.}}
\newcommand{\eg}{\notenglish{e.g.}}

\newcommand{\documentname}{document}
\newcommand{\sectionname}{Section}
\newcommand{\equationname}{equation}
\newcommand{\figurenames}{\figurename s}
\newcommand{\problemname}{Exercise}
\newcommand{\problemnames}{\problemname s}
\newcommand{\solutionname}{Solution}
\newcommand{\notename}{note}

\newcommand{\note}[1]{\endnote{#1}}
\def\enotesize{\normalsize}
\renewcommand{\thefootnote}{\fnsymbol{footnote}} % the ONE footnote needs this

\newcounter{problem}
\newenvironment{problem}{\paragraph{\problemname~\theproblem:}\refstepcounter{problem}}{}
\newcommand{\affil}[1]{{\footnotesize\textsl{#1}}}

% matrix stuff
\newcommand{\mmatrix}[1]{\boldsymbol{#1}}
\newcommand{\inverse}[1]{{#1}^{-1}}
\newcommand{\transpose}[1]{{#1}^{\scriptscriptstyle \top}}
\newcommand{\mA}{\mmatrix{A}}
\newcommand{\mAT}{\transpose{\mA}}
\newcommand{\mC}{\mmatrix{C}}
\newcommand{\mCinv}{\inverse{\mC}}
\newcommand{\mQ}{\mmatrix{Q}}
\newcommand{\mS}{\mmatrix{S}}
\newcommand{\mX}{\mmatrix{X}}
\newcommand{\mY}{\mmatrix{Y}}
\newcommand{\mYT}{\transpose{\mY}}
\newcommand{\mZ}{\mmatrix{Z}}
\newcommand{\vhat}{\mmatrix{\hat{v}}}

% parameter vectors
\newcommand{\parametervector}[1]{\mmatrix{#1}}
\newcommand{\pvtheta}{\parametervector{\theta}}

% set stuff
\newcommand{\setofall}[3]{\{{#1}\}_{{#2}}^{{#3}}}
\newcommand{\allq}{\setofall{q_i}{i=1}{N}}
\newcommand{\allx}{\setofall{x_i}{i=1}{N}}
\newcommand{\ally}{\setofall{y_i}{i=1}{N}}
\newcommand{\allxy}{\setofall{x_i,y_i}{i=1}{N}}
\newcommand{\allsigmay}{\setofall{\sigma_{yi}^2}{i=1}{N}}
\newcommand{\allS}{\setofall{\mS_i}{i=1}{N}}

% other random multiply used math symbols
\renewcommand{\d}{\mathrm{d}}
\newcommand{\mean}[1]{\left<{#1}\right>}
\newcommand{\like}{\mathscr{L}}


% header stuff
\renewcommand{\MakeUppercase}[1]{#1}
\pagestyle{myheadings}
\renewcommand{\sectionmark}[1]{\markright{\thesection.~#1}}
\markboth{What is a model?}{}

\begin{document}
\thispagestyle{plain}\raggedbottom
\section*{Data analysis recipes:\ \\
  What is a model?\footnotemark}

\footnotetext{The \notename s begin on page~\pageref{note:first},
  including the license\note{\label{note:first} Copyright 2011, 2012,
    2013 by the author.  You may copy and distribute this document
    provided that you make no changes to it whatsoever.}  and the
  acknowledgements\note{It is a pleasure to thank Coryn Bailer-Jones
    (MPIA), Brendon Brewer (Auckland), Jo Bovy (IAS), Dustin Lang
    (CMU) and Sam Roweis (deceased) for valuable discussions.  This
    research was partially supported by NASA, the NSF, and the
    Alexander von Humboldt Foundation.  This research made use of the
    Python programming language and the open-source Python packages
    scipy, numpy, and matplotlib.}.}

\noindent
David~W.~Hogg\\
\affil{Center~for~Cosmology~and~Particle~Physics, Department~of~Physics, New~York~University}\\
\affil{Max-Planck-Institut f\"ur Astronomie, Heidelberg}

\begin{abstract}
  All the inference in this series of \documentnames\ involves
  probabilistic comparison of models and data.  Here we lay out our
  specific meaning of the word ``model''; the short summary is that a
  model is an approximate (\emph{always} approximate) but justified
  expression for (something proportional to) the probability of the
  data, usually as a function of model parameters.  This is also
  called a ``likelihood function'' in many contexts.  Importantly in
  the physical sciences, likelihood function specification requires a
  quantitative description of the noise contributing to the data.
  Additional power comes to models in which it is possible to specify
  a prior probability density function over the parameters, and
  especially the \emph{nuisance} parameters; this power comes at the
  cost of the additional assumptions.  Many things conventionally
  described as ``models'' do not meet the criteria outlined here.
\end{abstract}

...Open with a few things people mean by the word ``model''.

...Introduce some (possibly fake, preferably real) data set to be ``modeled''.

...Introduce the parts of the model: Noise assumptions, mean assumptions,
likelihood function.

...Optimize the likelihood function and show what that looks like.

...Uncertainties based on the likelihood function alone.

...Note that certain parameters are nuisance parameters and you don't
want to make a statement about them.

...Introduce the idea of priors:  Marginalize out nuisance parameters.

...Optimize the marginalized likelihood and look at that.

...Now what if you have priors on everything of interest?  Now you can do MCMC.

...Note connections to other \documentnames\ in the series...  Note
connections to marginalized utility computation and decision theory...
And so on.

\clearpage
\markright{Notes}\theendnotes

\clearpage
\begin{thebibliography}{}\markright{References}
\bibitem[Bovy, Hogg, \& Roweis(2009)]{bovy}
  Bovy,~J., Hogg,~D.~W., \& Roweis, S.~T., 2009,
  Extreme deconvolution: inferring complete distribution functions from noisy, heterogeneous, and incomplete observations, 
  arXiv:0905.2979 [stat.ME]
\end{thebibliography}

\end{document}

\end{document}
