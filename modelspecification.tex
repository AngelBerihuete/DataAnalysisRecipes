% All content Copyright 2011 David W. Hogg (NYU).
% All rights reserved.

% to-do
% -----
% - Write!
% - Make consistent style file that deals with all the definitions, etc, for this and the other notes.

\documentclass[12pt,twoside]{article}
\usepackage{amssymb,amsmath,mathrsfs,,hogg_endnotes,natbib}

%%Figure caption
\makeatletter
\newsavebox{\tempbox}
\newcommand{\@makefigcaption}[2]{%
\vspace{10pt}{#1.--- #2\par}}%
\renewcommand{\figure}{\let\@makecaption\@makefigcaption\@float{figure}}
\makeatother

\setlength{\emergencystretch}{2em}%No overflow

\newcommand{\notenglish}[1]{\textsl{#1}}
\newcommand{\aposteriori}{\notenglish{a~posteriori}}
\newcommand{\apriori}{\notenglish{a~priori}}
\newcommand{\adhoc}{\notenglish{ad~hoc}}
\newcommand{\etal}{\notenglish{et al.}}
\newcommand{\eg}{\notenglish{e.g.}}

\newcommand{\documentname}{document}
\newcommand{\sectionname}{Section}
\newcommand{\equationname}{equation}
\newcommand{\problemname}{Exercise}
\newcommand{\solutionname}{Solution}
\newcommand{\notename}{note}

\newcounter{problem}
\newenvironment{problem}{\paragraph{\problemname~\theproblem:}\refstepcounter{problem}}{}
\newenvironment{comments}{\paragraph{\commentsname:}}{}
\newcommand{\affil}[1]{{\footnotesize\textsl{#1}}}

% matrix stuff
\newcommand{\mmatrix}[1]{\boldsymbol{#1}}
\newcommand{\inverse}[1]{{#1}^{-1}}
\newcommand{\transpose}[1]{{#1}^{\scriptscriptstyle \top}}
\newcommand{\mA}{\mmatrix{A}}
\newcommand{\mAT}{\transpose{\mA}}
\newcommand{\mC}{\mmatrix{C}}
\newcommand{\mCinv}{\inverse{\mC}}
\newcommand{\mE}{\mmatrix{E}}
\newcommand{\mQ}{\mmatrix{Q}}
\newcommand{\mS}{\mmatrix{S}}
\newcommand{\mX}{\mmatrix{X}}
\newcommand{\mY}{\mmatrix{Y}}
\newcommand{\mYT}{\transpose{\mY}}
\newcommand{\mZ}{\mmatrix{Z}}
\newcommand{\vhat}{\mmatrix{\hat{v}}}

% parameter vectors
\newcommand{\parametervector}[1]{\mmatrix{\vec{#1}}}
\newcommand{\pvtheta}{\parametervector{\theta}}

% set stuff
\newcommand{\setofall}[3]{\{{#1}\}_{{#2}}^{{#3}}}
\newcommand{\allq}{\setofall{q_i}{i=1}{N}}
\newcommand{\allx}{\setofall{x_i}{i=1}{N}}
\newcommand{\ally}{\setofall{y_i}{i=1}{N}}
\newcommand{\allxy}{\setofall{x_i,y_i}{i=1}{N}}
\newcommand{\allsigmay}{\setofall{\sigma_{yi}^2}{i=1}{N}}
\newcommand{\allC}{\setofall{\mC_i}{i=1}{N}}

% other random multiply used math symbols
\renewcommand{\d}{\mathrm{d}}
\newcommand{\like}{\mathscr{L}}
\newcommand{\Pbad}{P_{\mathrm{b}}}
\newcommand{\Ybad}{Y_{\mathrm{b}}}
\newcommand{\Vbad}{V_{\mathrm{b}}}
\newcommand{\bperp}{b_{\perp}}
\newcommand{\mean}[1]{\left<{#1}\right>}
\newcommand{\meanZ}{\mean{\mZ}}
\newcommand{\best}{\mathrm{best}}

% header stuff
\renewcommand{\MakeUppercase}[1]{#1}
\pagestyle{myheadings}
\renewcommand{\sectionmark}[1]{\markright{\thesection.~#1}}
\markboth{Model complexity}{}

% note stuff
\newcommand{\note}[1]{\endnote{#1}}
\def\enotesize{\normalsize}
\renewcommand{\thefootnote}{\fnsymbol{footnote}} % the ONE footnote needs this

\begin{document}
\thispagestyle{plain}\raggedbottom
\section*{Data analysis recipes:\ \\
  What is a model?\footnotemark}

\footnotetext{The \notename s begin on page~\pageref{note:first},
  including the license\note{\label{note:first} Copyright 2011 by the
    author.  You may copy and distribute this document provided that
    you make no changes to it whatsoever.}  and the
  acknowledgements\note{It is a pleasure to thank Coryn Bailer-Jones
    (MPIA), Jo Bovy (IAS), Dustin Lang (Princeton) and Sam Roweis
    (deceased) for valuable discussions.  This research was partially
    supported by NASA, the NSF, and the Alexander von Humboldt
    Foundation.  This research made use of the Python programming
    language and the open-source Python packages scipy, numpy, and
    matplotlib.}.}

\noindent
David~W.~Hogg\\
\affil{Center~for~Cosmology~and~Particle~Physics, Department~of~Physics, New~York~University}\\
\affil{Max-Planck-Institut f\"ur Astronomie, Heidelberg}
%% \\[1ex]
%% Jo~Bovy\\
%% \affil{Institute~for~Advanced~Study, Princeton}
%% \\[1ex]
%% Dustin~Lang\\
%% \affil{Princeton University Observatory}

\begin{abstract}
  All the inference in this (nonexistent, never to be completed) book
  involves the comparison of models and data.  Here we lay out our
  specific meaning of the word ``model''; the short summary is that a
  model is an approximate but justified expression for (something
  proportional to) the probability of the data, usually as a function
  of model parameters.  This is also called a ``likelihood function''
  in many contexts.  Importantly, model specification requires
  specification of the noise properties of the data.  Many things
  conventionally described as ``models'' do not meet the criteria
  outlined here.
\end{abstract}

Hello world!

\clearpage
\markright{Notes}\theendnotes

\clearpage
\begin{thebibliography}{}\markright{References}
\bibitem[Bovy, Hogg, \& Roweis(2009)]{bovy}
  Bovy,~J., Hogg,~D.~W., \& Roweis, S.~T., 2009,
  Extreme deconvolution: inferring complete distribution functions from noisy, heterogeneous, and incomplete observations, 
  arXiv:0905.2979 [stat.ME]
\end{thebibliography}

\end{document}

\end{document}
