% This document is part of the Data Analysis Recipes project.
% Copyright 2020 the author.

% to-do
% -----
% - make up a toy data set and make problems.
%   - make two different generative models for the toy data.
%   - should have bayes and frequentist options.
% - reformat in side-notes style like GaussianProductRefactor.
% - get BibTeX working like GPR.

\documentclass[12pt, letterpaper]{article}

\begin{document}\sloppy\sloppypar\raggedbottom\frenchspacing

\section*{Data Analysis Recipes:\\
  What is the uncertainty on my measurement?}

\textbf{David W. Hogg} \\
{\footnotesize New York University} \\
{\footnotesize Max-Planck-Institut f\"ur Astronomie} \\
{\footnotesize Flatiron Institute}

\paragraph{Abstract:}
Any measurement you make using data ought to be reported with an uncertainty
estimate (often called an ``error'' or an ``error bar'' unfortunately).
I discuss and compare methodologies for making such estimates.
The options availble to you depend on whether you have a generative model for your
data, with which either you can simulate your noisy data, or (even better)
compute probability densities for different data sets.
They also depend on whether you believe that model,
or believe what it implies for the moments of the noise distribution.
If you have a good generative model, information theory or data simulations can deliver
measurement uncertainties; if you don't they can often provide strong bounds.
Either way, bootstrap and jackknife methods provide well justified, empirical
alternatives that I recommend.
I also discuss the role of nuisance parameters in uncertainty estimation, and
the differences between Bayesian and frequentist approaches and interpretation.
I spend a bit of time on common issues and troubleshooting.
One important idea is that the circumstances in which an uncertainty can be estimated
precisely are rare: Even when you have a very precise measurement, you probably won't
know the uncertainty on that measurement with great precision.

\section{Measurements and uncertainties}

You have data. There's something you want to measure. You have many
options for making this measurement: You can come up with an
\textsl{estimator}, that transforms your data (through arithmetic
operations) into an estimate of the quantity you want to measure. You
can write down a likelihood function---a probability density function
(or pdf) for your data given the quantity you want to measure (and
maybe other nuisance parameters)---and you can optimize it. That
procedure will get you an estimator, but it would be a
\textsl{maximum-likelihood estimator}, which has some great
properties. Or you can write down, in addition to your likelihood
function, a set of prior pdfs over parameters and perform
\textsl{Bayesian inference}.  In each of these cases you will make
some kind of measurement, and in each of these cases you will be
expected to deliver that measurement with an associated \textsl{uncertainty}.

Hello world---syntactical reference.

Comments on interpretation; it's absurd! The frequentist interpretation is
absurd. But the Bayesian interpretation requires additonal assumptions.

Comments on terminology---should this be its own section?

\section{The standard errors on a least-square fit}

Hello world. Linear fit.

Non-linear fit.

\section{Information theory}

Connection to Cram\'er--Rao bound and Fisher information.

\section{Bayes}

Foo

You are only permitted to make one kind of uncertainty estimate in Bayes.

So you better believe that model really really well.

Or make it more baroque. (Good option for a problem / exercise!)

\section{Data simulation}

Hello world.

\section{Jackknife and bootstrap}

Hello world.

\section{Nuisance parameters}

Difference between marginalizing and profiling.

Identicality when it comes to the linear, Gaussian case.

What it looks like.

What it looks like in very nonlinear situations (like period fitting).

\section{Systematic error and theoretical uncertainty}

There is only bias and variance; nothing else.

\section{Common mistakes and troubleshooting}

About 68\,percent of your values should be outside one sigma. What do think
or do if that's not true?

The uncertainty on the mean vs the distribution of values.

My data aren't well fit by my model; what does that mean for my uncertainty
analysis?

Do I multiply my uncertainties up or do I add something in quadrature?

Do I multiply my uncertainties down?

\section{Discussion}

Hello world.

\end{document}
