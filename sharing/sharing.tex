% This file is part of the Data Analysis Recipes project.
% Copyright 2016 David W. Hogg (NYU).

% to-do
% -----
% - Make table
% - First draft

% style notes
% -----------
% - These documents are ``Chapters'', no?  Use \documentname.
% - [LaTeX] newline after every full stop for proper git diffing
% - [LaTeX] eqnarray not equation
% - \note{} right after word if you are endnoting a phrase,
%   but after the full stop if endnoting the sentence or paragraph.

\documentclass[12pt,twoside,pdftex]{article}
\usepackage{dar_endnotes}

\newcommand{\chaptertitle}{The costs and benefits of releasing your data and open-sourcing your code}
\newcommand{\shorttitle}{Releasing data and open-sourcing code}
\usepackage{amssymb}
\usepackage{amsmath}
\usepackage{mathrsfs}
\usepackage{natbib}
\usepackage{float}
\usepackage{graphicx}

% Figure captions
\makeatletter
\newsavebox{\tempbox}
\newcommand{\@makefigcaption}[2]{%
\vspace{10pt}{#1.--- #2\par}}%
\renewcommand{\figure}{\let\@makecaption\@makefigcaption\@float{figure}}
\makeatother

% No overflow
\setlength{\emergencystretch}{2em}

% Spacing -- Bring the Bringhurst
\linespread{1.08} % nearly 10/13
\setlength{\parindent}{1.08\baselineskip} % nearly square
\frenchspacing

% Typography
\newcommand{\documentname}{\textsl{Chapter}}
\newcommand{\sectionname}{Section}
\newcommand{\equationname}{equation}
\newcommand{\project}[1]{\textsl{#1}}
\newcommand{\foreign}[1]{\textsl{#1}}
\newcommand{\code}[1]{\texttt{\detokenize{#1}}}
\newenvironment{pseudocode}{\begin{ttfamily}\detokenize\obeylines}{}
\newcommand{\aposteriori}{\foreign{a~posteriori}}
\newcommand{\apriori}{\foreign{a~priori}}
\newcommand{\adhoc}{\foreign{ad~hoc}}
\newcommand{\etal}{\foreign{et al.}}
\newcommand{\eg}{\foreign{e.g.}}
\newcommand{\vs}{\foreign{vs.}}
\newcommand{\affil}[1]{{\footnotesize\textsl{#1}}}
\newcommand{\acronym}[1]{{\small{#1}}}

% Notes
\newcommand{\notename}{note}
\newcommand{\note}[1]{\endnote{#1}}
\def\enotesize{\normalsize}
\renewcommand{\thefootnote}{\fnsymbol{footnote}} % the ONE footnote needs this

% Problems
\newcommand{\problemname}{Problem}
\newcounter{problem}
\newenvironment{problem}{\paragraph{\problemname~\theproblem:}%
\refstepcounter{problem}}{}

% Math stuff
\newcommand{\mmatrix}[1]{\boldsymbol{#1}}
\newcommand{\inverse}[1]{{#1}^{-1}}
\newcommand{\transpose}[1]{{#1}^{\scriptscriptstyle \top}}
\newcommand{\parametervector}[1]{\mmatrix{#1}}
\newcommand{\bvec}[1]{\mmatrix{#1}}
\newcommand{\setof}[1]{\{{#1}\}}
\newcommand{\dd}{\mathrm{d}}
\newcommand{\given}{\,|\,}
\newcommand{\mean}[1]{\left<{#1}\right>}

% Headers
\renewcommand{\MakeUppercase}[1]{#1}
\pagestyle{myheadings}
\renewcommand{\sectionmark}[1]{\markright{\thesection.~#1}}
\markboth{\shorttitle}{}

% References.
\usepackage{color,hyperref}
\definecolor{linkcolor}{rgb}{0,0,0.5}
\hypersetup{colorlinks=true,linkcolor=linkcolor,citecolor=linkcolor,
            filecolor=linkcolor,urlcolor=linkcolor}
\newcommand{\arxiv}[1]{\href{http://arxiv.org/abs/#1}{\textsl{arXiv}:#1}}
\newcommand{\doi}[1]{\href{http://doi.org/#1}{\textsc{doi}:#1}}
\newcommand{\isbn}[1]{\textsc{isbn:}#1}


\begin{document}\sloppy\sloppypar\raggedbottom\thispagestyle{plain}%
%
\section*{Data analysis recipes:\\ \chaptertitle\footnotemark}
%
\footnotetext{%
    The notes begin on page~\pageref{note:first}, including the
    license\note{\label{note:first}
        Copyright 2016 the author.
        This work is licensed under a
        \href{http://creativecommons.org/licenses/by-nc-nd/3.0/deed.en\_US}{%
            Creative Commons Attribution-NonCommercial-NoDerivs 3.0 Unported
            License}.}
    and the acknowledgements\note{%
        I would like to thank
          Dustin Lang (Toronto),
          Dan Foreman-Mackey (UW),
          and the entire astrophysics community
        for valuable advice and comments over the years.
        This project benefitted from various research grants, especially
          from NSF, NASA,
          and the The Moore--Sloan Data Science Environment.}
}

\noindent
David~W.~Hogg\\
\affil{Simons~Center~for~Data~Analysis,
       Simons~Foundation}\\
\affil{Center~for~Cosmology~and~Particle~Physics, Department~of~Physics,
       New~York~University}\\
\affil{Center~for~Data~Science,
       New~York~University}\\
\affil{Max-Planck-Institut~f\"ur~Astronomie, Heidelberg}\\[1ex]

\begin{abstract}
foo, bar
\end{abstract}

\section{Economic pragmatism}

LTFDFCF.

Going to treat data and code in parallel for most of what follows.

\section{Overview}

\begin{table}
\begin{tabular}{@{}p{0.48\textwidth}|p{0.48\textwidth}@{}} % note @{} entries to remove space on edges
\multicolumn{1}{c}{costs} & \multicolumn{1}{c}{benefits} \\
\hline \multicolumn{2}{c}{releasing data} \\ \hline
this & that \\
the other & whatever \\
etc & ibid \\
\hline \multicolumn{2}{c}{open-sourcing code} \\ \hline
foo & bar \\ 
what & who \\
\end{tabular}
\caption{Summary of costs and benefits. Each of these is discussed in
  the main text.\label{tab:summary}}
\end{table}

\section{Personal considerations}

Search! Hosting! Preservation!

Documentation burden (bad). Documentation pressure (good).

Exposure of the horror (bad). Encouragement to be less horrible (good).

\section{Team considerations}

\section{Ethical considerations}

You might expect me here to go all ``ethically, you must release your
data!''; I won't, because there are many situations in which it is in
fact \emph{unethical} to release data and code.

Coming back to the personal point about exposing the horror and
requirements to document: Both of these have ethical aspects: The more
we expose the horror, the more we make people comfortable with
sharing. The more we document, the more we help others with
reproducibility and with new science.

\section{Licensing}

Open-sourcing your code means more than just publishing it.

Licensing has not really been explored for data, at least not in most
scientific domains.

\clearpage
\markright{Notes}\theendnotes

\clearpage
\raggedright
\begin{thebibliography}{}\markright{References}
\bibitem[Foreman-Mackey \etal(2012)]{emcee}
  Foreman-Mackey,~D., Hogg,~D.~W., Lang,~D., \& Goodman,~J.\ 2012,
  \project{emcee}: The \acronym{MCMC} Hammer,
  \textit{Pubs.\ Astron.\ Soc.\ Pac.}\ \textbf{125} 306--312
  \doi{10.1086/670067}
  (also \arxiv{1202.3665}).
\end{thebibliography}

\end{document}
