% This file is part of the Data Analysis Recipes project.
% Copyright 2012 David W. Hogg (NYU)

% to-do items
% -----------
% - outline
% - write
% - distribute for comments
% - post to arXiv

\documentclass[12pt,twoside]{article}
\usepackage{amssymb,amsmath,mathrsfs,hogg_endnotes,natbib}
\usepackage{float,graphicx}
%%Figure caption
\makeatletter
\newsavebox{\tempbox}
\newcommand{\@makefigcaption}[2]{%
\vspace{10pt}{#1.--- #2\par}}%
\renewcommand{\figure}{\let\@makecaption\@makefigcaption\@float{figure}}
\makeatother

\newcommand{\exampleplot}[1]{%
\begin{center}%
\includegraphics[width=0.5\textwidth]{#1}%
\end{center}%
}
\newcommand{\exampleplottwo}[2]{%
\begin{center}%
\includegraphics[width=0.5\textwidth]{#1}%
\includegraphics[width=0.5\textwidth]{#2}%
\end{center}%
}

\setlength{\emergencystretch}{2em}%No overflow

\newcommand{\notenglish}[1]{\textsl{#1}}
\newcommand{\aposteriori}{\notenglish{a~posteriori}}
\newcommand{\apriori}{\notenglish{a~priori}}
\newcommand{\adhoc}{\notenglish{ad~hoc}}
\newcommand{\etal}{\notenglish{et al.}}
\newcommand{\eg}{\notenglish{e.g.}}

\newcommand{\documentname}{document}
\newcommand{\sectionname}{Section}
\newcommand{\equationname}{equation}
\newcommand{\figurenames}{\figurename s}
\newcommand{\problemname}{Exercise}
\newcommand{\problemnames}{\problemname s}
\newcommand{\solutionname}{Solution}
\newcommand{\notename}{note}

\newcommand{\note}[1]{\endnote{#1}}
\def\enotesize{\normalsize}
\renewcommand{\thefootnote}{\fnsymbol{footnote}} % the ONE footnote needs this

\newcounter{problem}
\newenvironment{problem}{\paragraph{\problemname~\theproblem:}\refstepcounter{problem}}{}
\newcommand{\affil}[1]{{\footnotesize\textsl{#1}}}

% matrix stuff
\newcommand{\mmatrix}[1]{\boldsymbol{#1}}
\newcommand{\inverse}[1]{{#1}^{-1}}
\newcommand{\transpose}[1]{{#1}^{\scriptscriptstyle \top}}
\newcommand{\mA}{\mmatrix{A}}
\newcommand{\mAT}{\transpose{\mA}}
\newcommand{\mC}{\mmatrix{C}}
\newcommand{\mCinv}{\inverse{\mC}}
\newcommand{\mQ}{\mmatrix{Q}}
\newcommand{\mS}{\mmatrix{S}}
\newcommand{\mX}{\mmatrix{X}}
\newcommand{\mY}{\mmatrix{Y}}
\newcommand{\mYT}{\transpose{\mY}}
\newcommand{\mZ}{\mmatrix{Z}}
\newcommand{\vhat}{\mmatrix{\hat{v}}}

% parameter vectors
\newcommand{\parametervector}[1]{\mmatrix{#1}}
\newcommand{\pvtheta}{\parametervector{\theta}}

% set stuff
\newcommand{\setofall}[3]{\{{#1}\}_{{#2}}^{{#3}}}
\newcommand{\allq}{\setofall{q_i}{i=1}{N}}
\newcommand{\allx}{\setofall{x_i}{i=1}{N}}
\newcommand{\ally}{\setofall{y_i}{i=1}{N}}
\newcommand{\allxy}{\setofall{x_i,y_i}{i=1}{N}}
\newcommand{\allsigmay}{\setofall{\sigma_{yi}^2}{i=1}{N}}
\newcommand{\allS}{\setofall{\mS_i}{i=1}{N}}

% other random multiply used math symbols
\renewcommand{\d}{\mathrm{d}}
\newcommand{\mean}[1]{\left<{#1}\right>}
\newcommand{\like}{\mathscr{L}}


% header stuff
\renewcommand{\MakeUppercase}[1]{#1}
\pagestyle{myheadings}
\renewcommand{\sectionmark}[1]{\markright{\thesection.~#1}}
\markboth{Probabilistic graphical models}{Introduction}

\begin{document}
\thispagestyle{plain}\raggedbottom\sloppy\sloppypar
\section*{Data analysis recipes:\\
  Probabilistic graphical models\footnotemark}

\footnotetext{%
  The \notename s begin on page~\pageref{note:first}, including the
  license\note{\label{note:first}%
    THIS IS A DRAFT DATED 2012-09-13 AND IS NOT FOR DISTRIBUTION.  WHEN IT IS DONE WE WILL SAY:
    ``
    Copyright 2012 the authors.  You may copy and distribute this
    document provided that you make no changes to it whatsoever.
    ''}
  and the acknowledgements\note{%
    It is a pleasure to thank
      Stefan Harmeling (MPI-IS),
      Phil Marshall (Oxford),
      Iain Murray (Edinburgh), and
      Hans-Walter Rix (MPIA)
    for discussions and comments.  This
    research was partially supported by the US National Aeronautics
    and Space Administration and National Science Foundation.}}

\noindent
David~W.~Hogg\\
\affil{Center~for~Cosmology~and~Particle~Physics, Department~of~Physics, New York University}\\
\and~\affil{Max-Planck-Institut f\"ur Astronomie, Heidelberg}
\\[1ex]
Dan Foreman-Mackey\\
\affil{Center~for~Cosmology~and~Particle~Physics, Department~of~Physics, New York University}

\begin{abstract}
We present an introduction to probabilistic graphical models (PGMs;
also sometimes called ``directed acyclic graphs'') in the context of
building models of scientific data sets.  We focus on their use as a
conceptual tool, helping to clarify and express complicated
probabilistic expressions or really assumptions about complex,
hierarchical probability distributons.  PGMs are most useful for
describing generative models, in that they can be used to both
describe the probability expressions for those models, but also the
causal relationships that are being assumed (or inferred).  PGMs, in
themselves, do not perform or speed calculations; they also aren't, in
themselves, methods; they act to express unambiguously probabilistic
and causal relationships.
\end{abstract}

\noindent
Yo.

uh huh.

\begin{problem}
what is up?
\end{problem}

\clearpage
\markright{Notes}\theendnotes

\clearpage
\begin{thebibliography}{}\markright{References}\raggedright
\bibitem[Hogg \etal(2010a)]{straightline}
  Hogg,~D.~W., Bovy,~J., \& Lang,~D., 2010a,
  ``Data analysis recipes:\ Fitting a model to data'', arXiv:1008.4686
\bibitem[Hogg \etal(2010b)]{eccentricity}
  Hogg,~D.~W., Myers,~A.~D., \& Bovy,~J., 2010b,
  ``Inferring the eccentricity distribution'', arXiv:1008.4146
\end{thebibliography}

\end{document}
